\documentclass{beamer}
\usepackage{graphicx}
\usepackage{amssymb}
\usepackage{graphviz}
\usepackage{tikz}
%\usepackage{auto-pst-pdf}
\usepackage{dot2texi}
\usepackage{listings}

\pgfdeclarelayer{background}
\pgfdeclarelayer{foreground}
\pgfsetlayers{background,main,foreground}

\begin{document}
\def \SCoP {SCoP}
\def \GCC {GCC}
\def \LLVM {LLVM}
\def \SESE {SESE}
\def \CFG {CFG}
\def \SSA {SSA}
\def \scev {scev}

\title{\SCoP{} Detection: A Fast Algorithm for Industrial Compilers}
\author{Sebastian Pop and Aditya Kumar}
\institute{SARC: Samsung Austin R\&D Center}
\date{Jan 19, 2016}
%\date{\today}

\definecolor{myblue}{rgb}{0.0, 0.0, 0.5}
\definecolor{myred}{rgb}{0.5, 0.0, 0.0}
\definecolor{mygreen}{rgb}{0.0, 0.5, 0.0}
\lstset{language=C++,
  basicstyle=\ttfamily,
  keywordstyle=\color{myblue}\ttfamily,
  stringstyle=\color{myred}\ttfamily,
  commentstyle=\color{mygreen}\ttfamily,
  morecomment=[l][\color{magenta}]{\#}
}
\addtobeamertemplate{navigation symbols}{}{%
    \usebeamerfont{footline}%
    \usebeamercolor[fg]{footline}%
    \hspace{1em}%
    \insertframenumber/\inserttotalframenumber
}

\frame{\titlepage}

\frame{\frametitle{Polyhedral compilation in industrial compilers}
\begin{itemize}
\item Goal: enable isl scheduler in GCC at -O3
  \vspace{5mm}
  \pause
\item search loops that can benefit from polyhedral compilation
\item minimal overhead: search as fast as possible
\item only use existing analysis information
\item use the right abstract representation
\end{itemize}
}

\frame{\frametitle{What is a SCoP?}
  Regions of code that can be represented in the Polyhedral Model.
  \begin{itemize}
  \item SCoPs = Static Control Parts
    \pause
  \item ACLs = Affine Control Loops
  \item PWACs = Parts With Affine Control
  \end{itemize}
}

\begin{frame}[fragile]{Step 1: accept natural loops}
  \begin{columns}[T,onlytextwidth] % align columns
    \column{.2\textwidth}
    \begin{block}{\small Natural loop}
      \begin{tikzpicture}[scale=0.4]
        \begin{dot2tex}[dot,tikz,codeonly,styleonly]
          digraph G {
            {rank = same; a; x;}
            e -> a -> x;
            a -> b -> a [fontcolor=red];
          }
        \end{dot2tex}
      \end{tikzpicture}

      {\small \color{myblue} maybe SCoP}
    \end{block}
    \pause
    \column{.2\textwidth}
    \begin{block}{\small Nested loops}
      \begin{tikzpicture}[scale=0.4]
        \begin{dot2tex}[dot,tikz,codeonly,styleonly]
          digraph G {
            {rank = same; a; x;}
            e -> a -> x;
            a -> b -> d -> a;
            b -> c -> b;
          }
        \end{dot2tex}
      \end{tikzpicture}

      {\small \color{myblue} maybe SCoP}
    \end{block}
    \pause
    \column{.2\textwidth}
    \begin{block}{\small Irreducible}
      \begin{tikzpicture}[scale=0.4]
        \begin{dot2tex}[dot,tikz,codeonly,styleonly]
          digraph G {
            {rank = same; a; x;}
            e -> a -> x;
            a -> b -> a;
            a -> c -> a;
          }
        \end{dot2tex}
      \end{tikzpicture}

      {\small \color{myred} not a SCoP: ambiguous iteration order}
    \end{block}
    \pause
    \column{.2\textwidth}
    \begin{block}{\small Irreducible}
      \begin{tikzpicture}[scale=0.4]
        \begin{dot2tex}[dot,tikz,codeonly,styleonly]
          digraph G {
            {rank = same; a; x;}
            e -> a -> x;
            a -> b -> a;
            b -> c -> b;
          }
        \end{dot2tex}
      \end{tikzpicture}

      {\small \color{myred} not a SCoP: ambiguous iteration order}
    \end{block}
  \end{columns}
\end{frame}

\begin{frame}[fragile]{Natural Loop Tree}
  \begin{columns}[T,onlytextwidth] % align columns
    \column{0.7\textwidth}
    \begin{lstlisting}
int foo(int N)
{
 int i, j, k;
 for(i=0; i<N; ++i){//Loop1
  stmt1;
  for (j=0; j<N; ++j)//Loop2
   stmt2;
  for (k=0; k<N; ++k)//Loop3
   stmt3;
 }
}
    \end{lstlisting}
    \pause
    \column{.3\textwidth}
    \begin{tikzpicture}[scale=0.7]
      % set node style
      \tikzstyle{n} = [draw,shape=circle,minimum size=2em,
        inner sep=0pt,fill=red!20]
      \begin{dot2tex}[dot,tikz,codeonly,styleonly,options=-s -tmath]
        digraph G  {
          node [style="n"];
          {rank = same; Loop_2; Loop_3;}
          Function -> Loop_1 [label=" inner"];
          Loop_1 -> Loop_2 [label=" inner", orientation=90];
          Loop_2 -> Loop_3 [label="next"];
        }
      \end{dot2tex}
    \end{tikzpicture}
  \end{columns}
\end{frame}

%% \begin{frame}[fragile]{Step 2: accept structured control flow}
%%   \begin{columns}[T,onlytextwidth] % align columns
%%     \column{.3\textwidth}
%%     \begin{block}{\small simple condition}
%%       \begin{tikzpicture}[scale=0.4]
%%         \begin{dot2tex}[dot,tikz,codeonly,styleonly]
%%           digraph G {
%%             e -> c -> a ->x;
%%             c -> b -> x;
%%           }
%%         \end{dot2tex}
%%       \end{tikzpicture}

%%       {\small \color{myblue} maybe SCoP}
%%     \end{block}
%%     \pause
%%     \column{.3\textwidth}
%%     \begin{block}{\small nested conditions}
%%       \begin{tikzpicture}[scale=0.4]
%%         \begin{dot2tex}[dot,tikz,codeonly,styleonly]
%%           digraph G {
%%             e -> c -> b -> x;
%%             c -> a -> f -> x;
%%             a -> d -> x;
%%           }
%%         \end{dot2tex}
%%       \end{tikzpicture}

%%       {\small \color{myblue} maybe SCoP}
%%     \end{block}
%%     \pause
%%     \column{.3\textwidth}
%%     \begin{block}{\small unstructured}
%%       \begin{tikzpicture}[scale=0.4]
%%         \begin{dot2tex}[dot,tikz,codeonly,styleonly]
%%           digraph G {
%%             e -> c -> b -> d -> x;
%%             c -> a -> f -> x;
%%             a -> d;
%%           }
%%         \end{dot2tex}
%%       \end{tikzpicture}

%%       {\small \color{myred} not a SCoP: control dependences are hard}
%%     \end{block}
%%   \end{columns}
%% \end{frame}

\begin{frame}[fragile]{Step 2: check for side-effects}
  \begin{itemize}
  \item function calls
  \item inline assembly
  \item volatile operations
  \end{itemize}
\end{frame}

\begin{frame}[fragile]{Step 3: affine scalar evolutions}
  \begin{columns}[T,onlytextwidth] % align columns
    \column{.6\textwidth}
    \begin{block}{\small Linear}
      \begin{lstlisting}
i0 = phi_l1(0, i1)
// i0={0,+,1}_l1
i1 = i0 + 1
// i1={1,+,1}_l1
      \end{lstlisting}

      {\small \color{myblue} maybe SCoP}
    \end{block}
    \pause
    \begin{block}{\small Non-linear} 
      \begin{lstlisting}
j2 = phi_l1(3, j3)
j3 = j2 + i1
// j2={3,+,{1,+,1}_l1}_l1
      \end{lstlisting}
      {\small \color{myred} not an ACL: polynomial of degree 2}
    \end{block}
    \pause
    \column{.4\textwidth}
    \begin{block}{\small Non-linear} 
      \begin{lstlisting}
k4 = phi_l2(4, k5)
k5 = k4 * 2
// k4={4,*,2}_l2
      \end{lstlisting}
      {\small \color{myred} not an ACL: exponential}
    \end{block}

    \vspace{.5cm}
    \pause
    \begin{block}{\small analyzed expressions}
      \begin{itemize}
      \item branch conditions
      \item memory accesses
      \end{itemize}
    \end{block}
  \end{columns}
\end{frame}

\begin{frame}[fragile]{Step 4: delinearize memory access functions}
  \begin{columns}[T,onlytextwidth] % align columns
    \column{.6\textwidth}
    \begin{block}{\small Linear access functions}
      \begin{lstlisting}
A[100*i + 400*j]
B[i][j]
      \end{lstlisting}
      {\small \color{myblue} can represent in isl}
    \end{block}

    \vspace{.7cm}
    \pause
    \begin{block}{\small Non-linear access functions} 
      \begin{lstlisting}
C[i*i]
D[4*N*M*i + 4*M*j + 4*k]
E[4*i*N + 4*j]
      \end{lstlisting}

      {\small \color{myred} cannot represent in isl}
    \end{block}
    \pause
    \column{.4\textwidth}
    \begin{block}{\small delinearization}
      \begin{itemize}
      \item recognize array multi-dimensions
      \item compute linear access functions
      \end{itemize}
    \end{block}
    \pause
    \begin{block}{\small delinearized access functions}
      \begin{lstlisting}
int D[][N][M];
D[i][j][k]

int E[][N];
E[i][j]
      \end{lstlisting}

      {\small \color{myblue} can represent in isl}
    \end{block}
  \end{columns}
\end{frame}

\begin{frame}[fragile]{Overall picture: SCoP detection}
  \begin{columns}[T,onlytextwidth] % align columns
    \column{.4\textwidth}
    \begin{tikzpicture}[scale=0.4]
      \tikzstyle{n} = [draw,shape=rectangle]
      \begin{dot2tex}[dot,tikz,codeonly,styleonly]
        digraph G {
          node [style="n"];
          "Natural loops"
          -> "no side-effects?"
          -> "affine branch conditions?"
          -> "affine memory accesses?"
          -> "SCoP";
        }
      \end{dot2tex}
    \end{tikzpicture}
    \pause
    \column{.5\textwidth}
    Required analyses:
    \begin{itemize}
    \item natural loops tree
    \item (post-)dominators tree
    \item alias analysis
    \item scalar evolution analysis
    \end{itemize}
  \end{columns}
\end{frame}

\frame{\frametitle{Detecting SCoPs by induction on Natural Loops Tree}
  \begin{itemize}
  \item Start with a loop in the natural loops tree \\
    rather than the root of the CFG
    \vspace{1cm}
    \pause
  \item Focus on structure of natural loops \\
    before the validity of each statement
  \end{itemize}
}

\begin{frame}[fragile]{Example: Induction on Natural Loops Tree}
  \begin{tikzpicture}[scale=0.8]
    % set node style
    \tikzstyle{n} = [draw,shape=circle,minimum size=2em,
      inner sep=0pt,fill=red!20]
    \begin{dot2tex}[dot,tikz,codeonly,styleonly,options=-s -tmath]
      digraph G  {
        node [style="n"];
        {rank = same; Loop_2; Loop_3;}
        Function -> Loop_1 [label="inner"];
        Loop_1 -> Loop_2 [label="inner", orientation=90];
        Loop_2 -> Loop_3 [label="next"];
      }
    \end{dot2tex}
    \begin{pgfonlayer}{background}
      \pause
      \draw[rounded corners=2em,line width=3em,blue!30,cap=round]
      (Loop_2.center) -- (Loop_2.east);
      \pause
      \draw[rounded corners=2em,line width=3em,blue!30,cap=round]
      (Loop_3.west) -- (Loop_3.center);
      \pause
      \draw[rounded corners=2em,line width=3em,blue!30,cap=round]
      (Loop_2.center) -- (Loop_3.center);
      \pause
      \draw[rounded corners=2em,line width=3em,blue!30,cap=round]
      (Loop_1.center) -- (Loop_2.west) -- (Loop_3.east) -- (Loop_1.center);
      \fill[rounded corners=2em,line width=3em,blue!30,cap=round]
      (Loop_1.center) -- (Loop_2.center) -- (Loop_3.center);
    \end{pgfonlayer}
  \end{tikzpicture}
\end{frame}

\frame{\frametitle{Other implementations of SCoP Detection}
  \begin{itemize}
  \item Previous graphite SCoP detection based on CFG and DOM \\
    (misses the structure of loops)

    \vspace{1cm}
    \pause

  \item Polly's SCoP detection based on structure of SESE regions \\
    (full function body analysis even without interesting loops)

    \vspace{1cm}
    \pause

  \item Pet, Rose, other source-to-source compilers: SCoP detection \\
    based on the AST of a specific programming language
  \end{itemize}
}

\begin{frame}[fragile]{Experimental Results}
  \begin{columns}[T,onlytextwidth] % align columns
    \column{.4\textwidth}
    \begin{block}{\small Compilation time overhead}
      \resizebox{\linewidth}{!}{
        \begin{tabular}{|c|c|c|}
          \hline
          Benchmark  & Old \% & New \%  \\
          \hline
          Polybench  & $1.4$  & $1.9$   \\
          Tramp3d-v4 & $7.0$  & $0.3$   \\
          \GCC{} 6.0 & $0.24$ & $0.01$  \\
          \hline
        \end{tabular}
      }
    \end{block}

    \vspace{2cm} 
    \begin{block}{\small SCoP Metrics on Polybench}
      \resizebox{\linewidth}{!}{
        \begin{tabular}{|c|c|c|c|}
          \hline
          SCoP Metric   & Old  & New  & Polly  \\
          \hline
          Loops/\SCoP{} & 2.59 & 6.09 & 5.17   \\
          \hline
        \end{tabular}
      }
    \end{block}

    \column{.6\textwidth}
      \begin{center}
        \resizebox{.9\linewidth}{!}{% GNUPLOT: LaTeX picture
\setlength{\unitlength}{0.240900pt}
\ifx\plotpoint\undefined\newsavebox{\plotpoint}\fi
\sbox{\plotpoint}{\rule[-0.200pt]{0.400pt}{0.400pt}}%
\begin{picture}(1500,900)(0,0)
\sbox{\plotpoint}{\rule[-0.200pt]{0.400pt}{0.400pt}}%
\put(171.0,131.0){\rule[-0.200pt]{4.818pt}{0.400pt}}
\put(151,131){\makebox(0,0)[r]{ 0}}
\put(1419.0,131.0){\rule[-0.200pt]{4.818pt}{0.400pt}}
\put(171.0,277.0){\rule[-0.200pt]{4.818pt}{0.400pt}}
\put(151,277){\makebox(0,0)[r]{ 50}}
\put(1419.0,277.0){\rule[-0.200pt]{4.818pt}{0.400pt}}
\put(171.0,422.0){\rule[-0.200pt]{4.818pt}{0.400pt}}
\put(151,422){\makebox(0,0)[r]{ 100}}
\put(1419.0,422.0){\rule[-0.200pt]{4.818pt}{0.400pt}}
\put(171.0,568.0){\rule[-0.200pt]{4.818pt}{0.400pt}}
\put(151,568){\makebox(0,0)[r]{ 150}}
\put(1419.0,568.0){\rule[-0.200pt]{4.818pt}{0.400pt}}
\put(171.0,713.0){\rule[-0.200pt]{4.818pt}{0.400pt}}
\put(151,713){\makebox(0,0)[r]{ 200}}
\put(1419.0,713.0){\rule[-0.200pt]{4.818pt}{0.400pt}}
\put(171.0,859.0){\rule[-0.200pt]{4.818pt}{0.400pt}}
\put(151,859){\makebox(0,0)[r]{ 250}}
\put(1419.0,859.0){\rule[-0.200pt]{4.818pt}{0.400pt}}
\put(171.0,131.0){\rule[-0.200pt]{0.400pt}{4.818pt}}
\put(171,90){\makebox(0,0){ 0}}
\put(171.0,839.0){\rule[-0.200pt]{0.400pt}{4.818pt}}
\put(298.0,131.0){\rule[-0.200pt]{0.400pt}{4.818pt}}
\put(298,90){\makebox(0,0){ 50}}
\put(298.0,839.0){\rule[-0.200pt]{0.400pt}{4.818pt}}
\put(425.0,131.0){\rule[-0.200pt]{0.400pt}{4.818pt}}
\put(425,90){\makebox(0,0){ 100}}
\put(425.0,839.0){\rule[-0.200pt]{0.400pt}{4.818pt}}
\put(551.0,131.0){\rule[-0.200pt]{0.400pt}{4.818pt}}
\put(551,90){\makebox(0,0){ 150}}
\put(551.0,839.0){\rule[-0.200pt]{0.400pt}{4.818pt}}
\put(678.0,131.0){\rule[-0.200pt]{0.400pt}{4.818pt}}
\put(678,90){\makebox(0,0){ 200}}
\put(678.0,839.0){\rule[-0.200pt]{0.400pt}{4.818pt}}
\put(805.0,131.0){\rule[-0.200pt]{0.400pt}{4.818pt}}
\put(805,90){\makebox(0,0){ 250}}
\put(805.0,839.0){\rule[-0.200pt]{0.400pt}{4.818pt}}
\put(932.0,131.0){\rule[-0.200pt]{0.400pt}{4.818pt}}
\put(932,90){\makebox(0,0){ 300}}
\put(932.0,839.0){\rule[-0.200pt]{0.400pt}{4.818pt}}
\put(1059.0,131.0){\rule[-0.200pt]{0.400pt}{4.818pt}}
\put(1059,90){\makebox(0,0){ 350}}
\put(1059.0,839.0){\rule[-0.200pt]{0.400pt}{4.818pt}}
\put(1185.0,131.0){\rule[-0.200pt]{0.400pt}{4.818pt}}
\put(1185,90){\makebox(0,0){ 400}}
\put(1185.0,839.0){\rule[-0.200pt]{0.400pt}{4.818pt}}
\put(1312.0,131.0){\rule[-0.200pt]{0.400pt}{4.818pt}}
\put(1312,90){\makebox(0,0){ 450}}
\put(1312.0,839.0){\rule[-0.200pt]{0.400pt}{4.818pt}}
\put(1439.0,131.0){\rule[-0.200pt]{0.400pt}{4.818pt}}
\put(1439,90){\makebox(0,0){ 500}}
\put(1439.0,839.0){\rule[-0.200pt]{0.400pt}{4.818pt}}
\put(171.0,131.0){\rule[-0.200pt]{0.400pt}{175.375pt}}
\put(171.0,131.0){\rule[-0.200pt]{305.461pt}{0.400pt}}
\put(1439.0,131.0){\rule[-0.200pt]{0.400pt}{175.375pt}}
\put(171.0,859.0){\rule[-0.200pt]{305.461pt}{0.400pt}}
\put(30,495){\makebox(0,0){Speedup}}
\put(805,29){\makebox(0,0){Files of GCC}}
\put(171.0,131.0){\rule[-0.200pt]{0.400pt}{4.577pt}}
\put(174.0,131.0){\rule[-0.200pt]{0.400pt}{5.782pt}}
\put(176.0,131.0){\rule[-0.200pt]{0.400pt}{6.263pt}}
\put(179.0,131.0){\rule[-0.200pt]{0.400pt}{6.263pt}}
\put(181.0,131.0){\rule[-0.200pt]{0.400pt}{6.263pt}}
\put(184.0,131.0){\rule[-0.200pt]{0.400pt}{6.504pt}}
\put(186.0,131.0){\rule[-0.200pt]{0.400pt}{6.986pt}}
\put(189.0,131.0){\rule[-0.200pt]{0.400pt}{6.986pt}}
\put(191.0,131.0){\rule[-0.200pt]{0.400pt}{7.709pt}}
\put(194.0,131.0){\rule[-0.200pt]{0.400pt}{7.709pt}}
\put(196.0,131.0){\rule[-0.200pt]{0.400pt}{7.950pt}}
\put(199.0,131.0){\rule[-0.200pt]{0.400pt}{8.191pt}}
\put(201.0,131.0){\rule[-0.200pt]{0.400pt}{8.191pt}}
\put(204.0,131.0){\rule[-0.200pt]{0.400pt}{8.191pt}}
\put(207.0,131.0){\rule[-0.200pt]{0.400pt}{8.431pt}}
\put(209.0,131.0){\rule[-0.200pt]{0.400pt}{8.431pt}}
\put(212.0,131.0){\rule[-0.200pt]{0.400pt}{8.672pt}}
\put(214.0,131.0){\rule[-0.200pt]{0.400pt}{8.913pt}}
\put(217.0,131.0){\rule[-0.200pt]{0.400pt}{9.154pt}}
\put(219.0,131.0){\rule[-0.200pt]{0.400pt}{9.154pt}}
\put(222.0,131.0){\rule[-0.200pt]{0.400pt}{9.154pt}}
\put(224.0,131.0){\rule[-0.200pt]{0.400pt}{9.154pt}}
\put(227.0,131.0){\rule[-0.200pt]{0.400pt}{9.395pt}}
\put(229.0,131.0){\rule[-0.200pt]{0.400pt}{9.636pt}}
\put(232.0,131.0){\rule[-0.200pt]{0.400pt}{9.636pt}}
\put(234.0,131.0){\rule[-0.200pt]{0.400pt}{9.636pt}}
\put(237.0,131.0){\rule[-0.200pt]{0.400pt}{9.636pt}}
\put(239.0,131.0){\rule[-0.200pt]{0.400pt}{9.877pt}}
\put(242.0,131.0){\rule[-0.200pt]{0.400pt}{10.118pt}}
\put(245.0,131.0){\rule[-0.200pt]{0.400pt}{10.118pt}}
\put(247.0,131.0){\rule[-0.200pt]{0.400pt}{10.359pt}}
\put(250.0,131.0){\rule[-0.200pt]{0.400pt}{10.359pt}}
\put(252.0,131.0){\rule[-0.200pt]{0.400pt}{10.600pt}}
\put(255.0,131.0){\rule[-0.200pt]{0.400pt}{10.600pt}}
\put(257.0,131.0){\rule[-0.200pt]{0.400pt}{10.600pt}}
\put(260.0,131.0){\rule[-0.200pt]{0.400pt}{10.840pt}}
\put(262.0,131.0){\rule[-0.200pt]{0.400pt}{11.081pt}}
\put(265.0,131.0){\rule[-0.200pt]{0.400pt}{11.081pt}}
\put(267.0,131.0){\rule[-0.200pt]{0.400pt}{11.081pt}}
\put(270.0,131.0){\rule[-0.200pt]{0.400pt}{11.081pt}}
\put(272.0,131.0){\rule[-0.200pt]{0.400pt}{11.081pt}}
\put(275.0,131.0){\rule[-0.200pt]{0.400pt}{11.081pt}}
\put(278.0,131.0){\rule[-0.200pt]{0.400pt}{11.081pt}}
\put(280.0,131.0){\rule[-0.200pt]{0.400pt}{11.322pt}}
\put(283.0,131.0){\rule[-0.200pt]{0.400pt}{11.322pt}}
\put(285.0,131.0){\rule[-0.200pt]{0.400pt}{11.322pt}}
\put(288.0,131.0){\rule[-0.200pt]{0.400pt}{11.322pt}}
\put(290.0,131.0){\rule[-0.200pt]{0.400pt}{11.322pt}}
\put(293.0,131.0){\rule[-0.200pt]{0.400pt}{11.322pt}}
\put(295.0,131.0){\rule[-0.200pt]{0.400pt}{11.322pt}}
\put(298.0,131.0){\rule[-0.200pt]{0.400pt}{11.563pt}}
\put(300.0,131.0){\rule[-0.200pt]{0.400pt}{11.563pt}}
\put(303.0,131.0){\rule[-0.200pt]{0.400pt}{11.563pt}}
\put(305.0,131.0){\rule[-0.200pt]{0.400pt}{11.563pt}}
\put(308.0,131.0){\rule[-0.200pt]{0.400pt}{11.563pt}}
\put(310.0,131.0){\rule[-0.200pt]{0.400pt}{11.563pt}}
\put(313.0,131.0){\rule[-0.200pt]{0.400pt}{11.563pt}}
\put(316.0,131.0){\rule[-0.200pt]{0.400pt}{11.804pt}}
\put(318.0,131.0){\rule[-0.200pt]{0.400pt}{11.804pt}}
\put(321.0,131.0){\rule[-0.200pt]{0.400pt}{11.804pt}}
\put(323.0,131.0){\rule[-0.200pt]{0.400pt}{11.804pt}}
\put(326.0,131.0){\rule[-0.200pt]{0.400pt}{12.045pt}}
\put(328.0,131.0){\rule[-0.200pt]{0.400pt}{12.045pt}}
\put(331.0,131.0){\rule[-0.200pt]{0.400pt}{12.045pt}}
\put(333.0,131.0){\rule[-0.200pt]{0.400pt}{12.045pt}}
\put(336.0,131.0){\rule[-0.200pt]{0.400pt}{12.286pt}}
\put(338.0,131.0){\rule[-0.200pt]{0.400pt}{12.286pt}}
\put(341.0,131.0){\rule[-0.200pt]{0.400pt}{12.286pt}}
\put(343.0,131.0){\rule[-0.200pt]{0.400pt}{12.286pt}}
\put(346.0,131.0){\rule[-0.200pt]{0.400pt}{12.286pt}}
\put(349.0,131.0){\rule[-0.200pt]{0.400pt}{12.286pt}}
\put(351.0,131.0){\rule[-0.200pt]{0.400pt}{12.527pt}}
\put(354.0,131.0){\rule[-0.200pt]{0.400pt}{12.527pt}}
\put(356.0,131.0){\rule[-0.200pt]{0.400pt}{12.768pt}}
\put(359.0,131.0){\rule[-0.200pt]{0.400pt}{12.768pt}}
\put(361.0,131.0){\rule[-0.200pt]{0.400pt}{12.768pt}}
\put(364.0,131.0){\rule[-0.200pt]{0.400pt}{12.768pt}}
\put(366.0,131.0){\rule[-0.200pt]{0.400pt}{12.768pt}}
\put(369.0,131.0){\rule[-0.200pt]{0.400pt}{12.768pt}}
\put(371.0,131.0){\rule[-0.200pt]{0.400pt}{12.768pt}}
\put(374.0,131.0){\rule[-0.200pt]{0.400pt}{12.768pt}}
\put(376.0,131.0){\rule[-0.200pt]{0.400pt}{13.009pt}}
\put(379.0,131.0){\rule[-0.200pt]{0.400pt}{13.009pt}}
\put(381.0,131.0){\rule[-0.200pt]{0.400pt}{13.009pt}}
\put(384.0,131.0){\rule[-0.200pt]{0.400pt}{13.009pt}}
\put(387.0,131.0){\rule[-0.200pt]{0.400pt}{13.009pt}}
\put(389.0,131.0){\rule[-0.200pt]{0.400pt}{13.009pt}}
\put(392.0,131.0){\rule[-0.200pt]{0.400pt}{13.009pt}}
\put(394.0,131.0){\rule[-0.200pt]{0.400pt}{13.249pt}}
\put(397.0,131.0){\rule[-0.200pt]{0.400pt}{13.249pt}}
\put(399.0,131.0){\rule[-0.200pt]{0.400pt}{13.249pt}}
\put(402.0,131.0){\rule[-0.200pt]{0.400pt}{13.249pt}}
\put(404.0,131.0){\rule[-0.200pt]{0.400pt}{13.249pt}}
\put(407.0,131.0){\rule[-0.200pt]{0.400pt}{13.249pt}}
\put(409.0,131.0){\rule[-0.200pt]{0.400pt}{13.249pt}}
\put(412.0,131.0){\rule[-0.200pt]{0.400pt}{13.490pt}}
\put(414.0,131.0){\rule[-0.200pt]{0.400pt}{13.490pt}}
\put(417.0,131.0){\rule[-0.200pt]{0.400pt}{13.731pt}}
\put(420.0,131.0){\rule[-0.200pt]{0.400pt}{13.731pt}}
\put(422.0,131.0){\rule[-0.200pt]{0.400pt}{13.731pt}}
\put(425.0,131.0){\rule[-0.200pt]{0.400pt}{13.731pt}}
\put(427.0,131.0){\rule[-0.200pt]{0.400pt}{13.731pt}}
\put(430.0,131.0){\rule[-0.200pt]{0.400pt}{13.731pt}}
\put(432.0,131.0){\rule[-0.200pt]{0.400pt}{13.731pt}}
\put(435.0,131.0){\rule[-0.200pt]{0.400pt}{13.972pt}}
\put(437.0,131.0){\rule[-0.200pt]{0.400pt}{13.972pt}}
\put(440.0,131.0){\rule[-0.200pt]{0.400pt}{13.972pt}}
\put(442.0,131.0){\rule[-0.200pt]{0.400pt}{13.972pt}}
\put(445.0,131.0){\rule[-0.200pt]{0.400pt}{13.972pt}}
\put(447.0,131.0){\rule[-0.200pt]{0.400pt}{13.972pt}}
\put(450.0,131.0){\rule[-0.200pt]{0.400pt}{14.213pt}}
\put(452.0,131.0){\rule[-0.200pt]{0.400pt}{14.213pt}}
\put(455.0,131.0){\rule[-0.200pt]{0.400pt}{14.213pt}}
\put(458.0,131.0){\rule[-0.200pt]{0.400pt}{14.213pt}}
\put(460.0,131.0){\rule[-0.200pt]{0.400pt}{14.213pt}}
\put(463.0,131.0){\rule[-0.200pt]{0.400pt}{14.454pt}}
\put(465.0,131.0){\rule[-0.200pt]{0.400pt}{14.454pt}}
\put(468.0,131.0){\rule[-0.200pt]{0.400pt}{14.454pt}}
\put(470.0,131.0){\rule[-0.200pt]{0.400pt}{14.454pt}}
\put(473.0,131.0){\rule[-0.200pt]{0.400pt}{14.454pt}}
\put(475.0,131.0){\rule[-0.200pt]{0.400pt}{14.454pt}}
\put(478.0,131.0){\rule[-0.200pt]{0.400pt}{14.454pt}}
\put(480.0,131.0){\rule[-0.200pt]{0.400pt}{14.454pt}}
\put(483.0,131.0){\rule[-0.200pt]{0.400pt}{14.695pt}}
\put(485.0,131.0){\rule[-0.200pt]{0.400pt}{14.695pt}}
\put(488.0,131.0){\rule[-0.200pt]{0.400pt}{14.695pt}}
\put(491.0,131.0){\rule[-0.200pt]{0.400pt}{14.695pt}}
\put(493.0,131.0){\rule[-0.200pt]{0.400pt}{14.695pt}}
\put(496.0,131.0){\rule[-0.200pt]{0.400pt}{14.936pt}}
\put(498.0,131.0){\rule[-0.200pt]{0.400pt}{14.936pt}}
\put(501.0,131.0){\rule[-0.200pt]{0.400pt}{14.936pt}}
\put(503.0,131.0){\rule[-0.200pt]{0.400pt}{14.936pt}}
\put(506.0,131.0){\rule[-0.200pt]{0.400pt}{14.936pt}}
\put(508.0,131.0){\rule[-0.200pt]{0.400pt}{15.177pt}}
\put(511.0,131.0){\rule[-0.200pt]{0.400pt}{15.177pt}}
\put(513.0,131.0){\rule[-0.200pt]{0.400pt}{15.177pt}}
\put(516.0,131.0){\rule[-0.200pt]{0.400pt}{15.177pt}}
\put(518.0,131.0){\rule[-0.200pt]{0.400pt}{15.177pt}}
\put(521.0,131.0){\rule[-0.200pt]{0.400pt}{15.177pt}}
\put(524.0,131.0){\rule[-0.200pt]{0.400pt}{15.418pt}}
\put(526.0,131.0){\rule[-0.200pt]{0.400pt}{15.418pt}}
\put(529.0,131.0){\rule[-0.200pt]{0.400pt}{15.418pt}}
\put(531.0,131.0){\rule[-0.200pt]{0.400pt}{15.658pt}}
\put(534.0,131.0){\rule[-0.200pt]{0.400pt}{15.658pt}}
\put(536.0,131.0){\rule[-0.200pt]{0.400pt}{15.658pt}}
\put(539.0,131.0){\rule[-0.200pt]{0.400pt}{15.658pt}}
\put(541.0,131.0){\rule[-0.200pt]{0.400pt}{15.658pt}}
\put(544.0,131.0){\rule[-0.200pt]{0.400pt}{15.899pt}}
\put(546.0,131.0){\rule[-0.200pt]{0.400pt}{15.899pt}}
\put(549.0,131.0){\rule[-0.200pt]{0.400pt}{15.899pt}}
\put(551.0,131.0){\rule[-0.200pt]{0.400pt}{15.899pt}}
\put(554.0,131.0){\rule[-0.200pt]{0.400pt}{15.899pt}}
\put(556.0,131.0){\rule[-0.200pt]{0.400pt}{15.899pt}}
\put(559.0,131.0){\rule[-0.200pt]{0.400pt}{16.140pt}}
\put(562.0,131.0){\rule[-0.200pt]{0.400pt}{16.140pt}}
\put(564.0,131.0){\rule[-0.200pt]{0.400pt}{16.140pt}}
\put(567.0,131.0){\rule[-0.200pt]{0.400pt}{16.140pt}}
\put(569.0,131.0){\rule[-0.200pt]{0.400pt}{16.140pt}}
\put(572.0,131.0){\rule[-0.200pt]{0.400pt}{16.140pt}}
\put(574.0,131.0){\rule[-0.200pt]{0.400pt}{16.140pt}}
\put(577.0,131.0){\rule[-0.200pt]{0.400pt}{16.381pt}}
\put(579.0,131.0){\rule[-0.200pt]{0.400pt}{16.381pt}}
\put(582.0,131.0){\rule[-0.200pt]{0.400pt}{16.381pt}}
\put(584.0,131.0){\rule[-0.200pt]{0.400pt}{16.381pt}}
\put(587.0,131.0){\rule[-0.200pt]{0.400pt}{16.381pt}}
\put(589.0,131.0){\rule[-0.200pt]{0.400pt}{16.622pt}}
\put(592.0,131.0){\rule[-0.200pt]{0.400pt}{16.622pt}}
\put(595.0,131.0){\rule[-0.200pt]{0.400pt}{16.622pt}}
\put(597.0,131.0){\rule[-0.200pt]{0.400pt}{16.622pt}}
\put(600.0,131.0){\rule[-0.200pt]{0.400pt}{16.622pt}}
\put(602.0,131.0){\rule[-0.200pt]{0.400pt}{16.622pt}}
\put(605.0,131.0){\rule[-0.200pt]{0.400pt}{16.622pt}}
\put(607.0,131.0){\rule[-0.200pt]{0.400pt}{16.622pt}}
\put(610.0,131.0){\rule[-0.200pt]{0.400pt}{16.863pt}}
\put(612.0,131.0){\rule[-0.200pt]{0.400pt}{16.863pt}}
\put(615.0,131.0){\rule[-0.200pt]{0.400pt}{17.104pt}}
\put(617.0,131.0){\rule[-0.200pt]{0.400pt}{17.104pt}}
\put(620.0,131.0){\rule[-0.200pt]{0.400pt}{17.104pt}}
\put(622.0,131.0){\rule[-0.200pt]{0.400pt}{17.104pt}}
\put(625.0,131.0){\rule[-0.200pt]{0.400pt}{17.104pt}}
\put(627.0,131.0){\rule[-0.200pt]{0.400pt}{17.104pt}}
\put(630.0,131.0){\rule[-0.200pt]{0.400pt}{17.104pt}}
\put(633.0,131.0){\rule[-0.200pt]{0.400pt}{17.104pt}}
\put(635.0,131.0){\rule[-0.200pt]{0.400pt}{17.345pt}}
\put(638.0,131.0){\rule[-0.200pt]{0.400pt}{17.345pt}}
\put(640.0,131.0){\rule[-0.200pt]{0.400pt}{17.345pt}}
\put(643.0,131.0){\rule[-0.200pt]{0.400pt}{17.586pt}}
\put(645.0,131.0){\rule[-0.200pt]{0.400pt}{17.586pt}}
\put(648.0,131.0){\rule[-0.200pt]{0.400pt}{17.586pt}}
\put(650.0,131.0){\rule[-0.200pt]{0.400pt}{17.586pt}}
\put(653.0,131.0){\rule[-0.200pt]{0.400pt}{17.586pt}}
\put(655.0,131.0){\rule[-0.200pt]{0.400pt}{17.586pt}}
\put(658.0,131.0){\rule[-0.200pt]{0.400pt}{17.827pt}}
\put(660.0,131.0){\rule[-0.200pt]{0.400pt}{17.827pt}}
\put(663.0,131.0){\rule[-0.200pt]{0.400pt}{17.827pt}}
\put(666.0,131.0){\rule[-0.200pt]{0.400pt}{17.827pt}}
\put(668.0,131.0){\rule[-0.200pt]{0.400pt}{17.827pt}}
\put(671.0,131.0){\rule[-0.200pt]{0.400pt}{17.827pt}}
\put(673.0,131.0){\rule[-0.200pt]{0.400pt}{17.827pt}}
\put(676.0,131.0){\rule[-0.200pt]{0.400pt}{18.067pt}}
\put(678.0,131.0){\rule[-0.200pt]{0.400pt}{18.067pt}}
\put(681.0,131.0){\rule[-0.200pt]{0.400pt}{18.067pt}}
\put(683.0,131.0){\rule[-0.200pt]{0.400pt}{18.308pt}}
\put(686.0,131.0){\rule[-0.200pt]{0.400pt}{18.308pt}}
\put(688.0,131.0){\rule[-0.200pt]{0.400pt}{18.549pt}}
\put(691.0,131.0){\rule[-0.200pt]{0.400pt}{18.549pt}}
\put(693.0,131.0){\rule[-0.200pt]{0.400pt}{18.549pt}}
\put(696.0,131.0){\rule[-0.200pt]{0.400pt}{18.549pt}}
\put(698.0,131.0){\rule[-0.200pt]{0.400pt}{18.549pt}}
\put(701.0,131.0){\rule[-0.200pt]{0.400pt}{18.790pt}}
\put(704.0,131.0){\rule[-0.200pt]{0.400pt}{18.790pt}}
\put(706.0,131.0){\rule[-0.200pt]{0.400pt}{18.790pt}}
\put(709.0,131.0){\rule[-0.200pt]{0.400pt}{18.790pt}}
\put(711.0,131.0){\rule[-0.200pt]{0.400pt}{18.790pt}}
\put(714.0,131.0){\rule[-0.200pt]{0.400pt}{18.790pt}}
\put(716.0,131.0){\rule[-0.200pt]{0.400pt}{18.790pt}}
\put(719.0,131.0){\rule[-0.200pt]{0.400pt}{18.790pt}}
\put(721.0,131.0){\rule[-0.200pt]{0.400pt}{18.790pt}}
\put(724.0,131.0){\rule[-0.200pt]{0.400pt}{19.031pt}}
\put(726.0,131.0){\rule[-0.200pt]{0.400pt}{19.272pt}}
\put(729.0,131.0){\rule[-0.200pt]{0.400pt}{19.272pt}}
\put(731.0,131.0){\rule[-0.200pt]{0.400pt}{19.272pt}}
\put(734.0,131.0){\rule[-0.200pt]{0.400pt}{19.272pt}}
\put(737.0,131.0){\rule[-0.200pt]{0.400pt}{19.513pt}}
\put(739.0,131.0){\rule[-0.200pt]{0.400pt}{19.513pt}}
\put(742.0,131.0){\rule[-0.200pt]{0.400pt}{19.513pt}}
\put(744.0,131.0){\rule[-0.200pt]{0.400pt}{19.513pt}}
\put(747.0,131.0){\rule[-0.200pt]{0.400pt}{19.754pt}}
\put(749.0,131.0){\rule[-0.200pt]{0.400pt}{19.754pt}}
\put(752.0,131.0){\rule[-0.200pt]{0.400pt}{19.754pt}}
\put(754.0,131.0){\rule[-0.200pt]{0.400pt}{19.754pt}}
\put(757.0,131.0){\rule[-0.200pt]{0.400pt}{19.754pt}}
\put(759.0,131.0){\rule[-0.200pt]{0.400pt}{19.754pt}}
\put(762.0,131.0){\rule[-0.200pt]{0.400pt}{19.754pt}}
\put(764.0,131.0){\rule[-0.200pt]{0.400pt}{19.995pt}}
\put(767.0,131.0){\rule[-0.200pt]{0.400pt}{19.995pt}}
\put(769.0,131.0){\rule[-0.200pt]{0.400pt}{19.995pt}}
\put(772.0,131.0){\rule[-0.200pt]{0.400pt}{19.995pt}}
\put(775.0,131.0){\rule[-0.200pt]{0.400pt}{19.995pt}}
\put(777.0,131.0){\rule[-0.200pt]{0.400pt}{20.236pt}}
\put(780.0,131.0){\rule[-0.200pt]{0.400pt}{20.236pt}}
\put(782.0,131.0){\rule[-0.200pt]{0.400pt}{20.236pt}}
\put(785.0,131.0){\rule[-0.200pt]{0.400pt}{20.236pt}}
\put(787.0,131.0){\rule[-0.200pt]{0.400pt}{20.236pt}}
\put(790.0,131.0){\rule[-0.200pt]{0.400pt}{20.236pt}}
\put(792.0,131.0){\rule[-0.200pt]{0.400pt}{20.236pt}}
\put(795.0,131.0){\rule[-0.200pt]{0.400pt}{20.476pt}}
\put(797.0,131.0){\rule[-0.200pt]{0.400pt}{20.476pt}}
\put(800.0,131.0){\rule[-0.200pt]{0.400pt}{20.476pt}}
\put(802.0,131.0){\rule[-0.200pt]{0.400pt}{20.476pt}}
\put(805.0,131.0){\rule[-0.200pt]{0.400pt}{20.476pt}}
\put(808.0,131.0){\rule[-0.200pt]{0.400pt}{20.476pt}}
\put(810.0,131.0){\rule[-0.200pt]{0.400pt}{20.717pt}}
\put(813.0,131.0){\rule[-0.200pt]{0.400pt}{20.717pt}}
\put(815.0,131.0){\rule[-0.200pt]{0.400pt}{20.717pt}}
\put(818.0,131.0){\rule[-0.200pt]{0.400pt}{20.717pt}}
\put(820.0,131.0){\rule[-0.200pt]{0.400pt}{20.717pt}}
\put(823.0,131.0){\rule[-0.200pt]{0.400pt}{20.717pt}}
\put(825.0,131.0){\rule[-0.200pt]{0.400pt}{20.958pt}}
\put(828.0,131.0){\rule[-0.200pt]{0.400pt}{20.958pt}}
\put(830.0,131.0){\rule[-0.200pt]{0.400pt}{20.958pt}}
\put(833.0,131.0){\rule[-0.200pt]{0.400pt}{20.958pt}}
\put(835.0,131.0){\rule[-0.200pt]{0.400pt}{20.958pt}}
\put(838.0,131.0){\rule[-0.200pt]{0.400pt}{20.958pt}}
\put(841.0,131.0){\rule[-0.200pt]{0.400pt}{21.199pt}}
\put(843.0,131.0){\rule[-0.200pt]{0.400pt}{21.199pt}}
\put(846.0,131.0){\rule[-0.200pt]{0.400pt}{21.440pt}}
\put(848.0,131.0){\rule[-0.200pt]{0.400pt}{21.440pt}}
\put(851.0,131.0){\rule[-0.200pt]{0.400pt}{21.440pt}}
\put(853.0,131.0){\rule[-0.200pt]{0.400pt}{21.440pt}}
\put(856.0,131.0){\rule[-0.200pt]{0.400pt}{21.440pt}}
\put(858.0,131.0){\rule[-0.200pt]{0.400pt}{21.440pt}}
\put(861.0,131.0){\rule[-0.200pt]{0.400pt}{21.681pt}}
\put(863.0,131.0){\rule[-0.200pt]{0.400pt}{21.681pt}}
\put(866.0,131.0){\rule[-0.200pt]{0.400pt}{21.681pt}}
\put(868.0,131.0){\rule[-0.200pt]{0.400pt}{21.681pt}}
\put(871.0,131.0){\rule[-0.200pt]{0.400pt}{21.922pt}}
\put(873.0,131.0){\rule[-0.200pt]{0.400pt}{21.922pt}}
\put(876.0,131.0){\rule[-0.200pt]{0.400pt}{21.922pt}}
\put(879.0,131.0){\rule[-0.200pt]{0.400pt}{21.922pt}}
\put(881.0,131.0){\rule[-0.200pt]{0.400pt}{21.922pt}}
\put(884.0,131.0){\rule[-0.200pt]{0.400pt}{21.922pt}}
\put(886.0,131.0){\rule[-0.200pt]{0.400pt}{21.922pt}}
\put(889.0,131.0){\rule[-0.200pt]{0.400pt}{21.922pt}}
\put(891.0,131.0){\rule[-0.200pt]{0.400pt}{21.922pt}}
\put(894.0,131.0){\rule[-0.200pt]{0.400pt}{21.922pt}}
\put(896.0,131.0){\rule[-0.200pt]{0.400pt}{22.163pt}}
\put(899.0,131.0){\rule[-0.200pt]{0.400pt}{22.163pt}}
\put(901.0,131.0){\rule[-0.200pt]{0.400pt}{22.163pt}}
\put(904.0,131.0){\rule[-0.200pt]{0.400pt}{22.163pt}}
\put(906.0,131.0){\rule[-0.200pt]{0.400pt}{22.163pt}}
\put(909.0,131.0){\rule[-0.200pt]{0.400pt}{22.404pt}}
\put(912.0,131.0){\rule[-0.200pt]{0.400pt}{22.404pt}}
\put(914.0,131.0){\rule[-0.200pt]{0.400pt}{22.404pt}}
\put(917.0,131.0){\rule[-0.200pt]{0.400pt}{22.404pt}}
\put(919.0,131.0){\rule[-0.200pt]{0.400pt}{22.404pt}}
\put(922.0,131.0){\rule[-0.200pt]{0.400pt}{22.404pt}}
\put(924.0,131.0){\rule[-0.200pt]{0.400pt}{22.404pt}}
\put(927.0,131.0){\rule[-0.200pt]{0.400pt}{22.404pt}}
\put(929.0,131.0){\rule[-0.200pt]{0.400pt}{22.404pt}}
\put(932.0,131.0){\rule[-0.200pt]{0.400pt}{22.645pt}}
\put(934.0,131.0){\rule[-0.200pt]{0.400pt}{22.645pt}}
\put(937.0,131.0){\rule[-0.200pt]{0.400pt}{22.645pt}}
\put(939.0,131.0){\rule[-0.200pt]{0.400pt}{22.645pt}}
\put(942.0,131.0){\rule[-0.200pt]{0.400pt}{22.645pt}}
\put(944.0,131.0){\rule[-0.200pt]{0.400pt}{22.645pt}}
\put(947.0,131.0){\rule[-0.200pt]{0.400pt}{22.645pt}}
\put(950.0,131.0){\rule[-0.200pt]{0.400pt}{22.645pt}}
\put(952.0,131.0){\rule[-0.200pt]{0.400pt}{22.885pt}}
\put(955.0,131.0){\rule[-0.200pt]{0.400pt}{22.885pt}}
\put(957.0,131.0){\rule[-0.200pt]{0.400pt}{22.885pt}}
\put(960.0,131.0){\rule[-0.200pt]{0.400pt}{22.885pt}}
\put(962.0,131.0){\rule[-0.200pt]{0.400pt}{22.885pt}}
\put(965.0,131.0){\rule[-0.200pt]{0.400pt}{23.126pt}}
\put(967.0,131.0){\rule[-0.200pt]{0.400pt}{23.126pt}}
\put(970.0,131.0){\rule[-0.200pt]{0.400pt}{23.126pt}}
\put(972.0,131.0){\rule[-0.200pt]{0.400pt}{23.126pt}}
\put(975.0,131.0){\rule[-0.200pt]{0.400pt}{23.126pt}}
\put(977.0,131.0){\rule[-0.200pt]{0.400pt}{23.126pt}}
\put(980.0,131.0){\rule[-0.200pt]{0.400pt}{23.126pt}}
\put(983.0,131.0){\rule[-0.200pt]{0.400pt}{23.367pt}}
\put(985.0,131.0){\rule[-0.200pt]{0.400pt}{23.608pt}}
\put(988.0,131.0){\rule[-0.200pt]{0.400pt}{23.608pt}}
\put(990.0,131.0){\rule[-0.200pt]{0.400pt}{23.608pt}}
\put(993.0,131.0){\rule[-0.200pt]{0.400pt}{23.608pt}}
\put(995.0,131.0){\rule[-0.200pt]{0.400pt}{23.608pt}}
\put(998.0,131.0){\rule[-0.200pt]{0.400pt}{23.849pt}}
\put(1000.0,131.0){\rule[-0.200pt]{0.400pt}{23.849pt}}
\put(1003.0,131.0){\rule[-0.200pt]{0.400pt}{23.849pt}}
\put(1005.0,131.0){\rule[-0.200pt]{0.400pt}{23.849pt}}
\put(1008.0,131.0){\rule[-0.200pt]{0.400pt}{24.090pt}}
\put(1010.0,131.0){\rule[-0.200pt]{0.400pt}{24.090pt}}
\put(1013.0,131.0){\rule[-0.200pt]{0.400pt}{24.090pt}}
\put(1015.0,131.0){\rule[-0.200pt]{0.400pt}{24.331pt}}
\put(1018.0,131.0){\rule[-0.200pt]{0.400pt}{24.331pt}}
\put(1021.0,131.0){\rule[-0.200pt]{0.400pt}{24.331pt}}
\put(1023.0,131.0){\rule[-0.200pt]{0.400pt}{24.331pt}}
\put(1026.0,131.0){\rule[-0.200pt]{0.400pt}{24.331pt}}
\put(1028.0,131.0){\rule[-0.200pt]{0.400pt}{24.331pt}}
\put(1031.0,131.0){\rule[-0.200pt]{0.400pt}{24.572pt}}
\put(1033.0,131.0){\rule[-0.200pt]{0.400pt}{24.572pt}}
\put(1036.0,131.0){\rule[-0.200pt]{0.400pt}{24.572pt}}
\put(1038.0,131.0){\rule[-0.200pt]{0.400pt}{24.572pt}}
\put(1041.0,131.0){\rule[-0.200pt]{0.400pt}{24.572pt}}
\put(1043.0,131.0){\rule[-0.200pt]{0.400pt}{24.572pt}}
\put(1046.0,131.0){\rule[-0.200pt]{0.400pt}{24.813pt}}
\put(1048.0,131.0){\rule[-0.200pt]{0.400pt}{24.813pt}}
\put(1051.0,131.0){\rule[-0.200pt]{0.400pt}{24.813pt}}
\put(1054.0,131.0){\rule[-0.200pt]{0.400pt}{24.813pt}}
\put(1056.0,131.0){\rule[-0.200pt]{0.400pt}{24.813pt}}
\put(1059.0,131.0){\rule[-0.200pt]{0.400pt}{24.813pt}}
\put(1061.0,131.0){\rule[-0.200pt]{0.400pt}{24.813pt}}
\put(1064.0,131.0){\rule[-0.200pt]{0.400pt}{25.054pt}}
\put(1066.0,131.0){\rule[-0.200pt]{0.400pt}{25.294pt}}
\put(1069.0,131.0){\rule[-0.200pt]{0.400pt}{25.294pt}}
\put(1071.0,131.0){\rule[-0.200pt]{0.400pt}{25.294pt}}
\put(1074.0,131.0){\rule[-0.200pt]{0.400pt}{25.535pt}}
\put(1076.0,131.0){\rule[-0.200pt]{0.400pt}{25.535pt}}
\put(1079.0,131.0){\rule[-0.200pt]{0.400pt}{25.535pt}}
\put(1081.0,131.0){\rule[-0.200pt]{0.400pt}{25.776pt}}
\put(1084.0,131.0){\rule[-0.200pt]{0.400pt}{25.776pt}}
\put(1086.0,131.0){\rule[-0.200pt]{0.400pt}{25.776pt}}
\put(1089.0,131.0){\rule[-0.200pt]{0.400pt}{25.776pt}}
\put(1092.0,131.0){\rule[-0.200pt]{0.400pt}{26.017pt}}
\put(1094.0,131.0){\rule[-0.200pt]{0.400pt}{26.017pt}}
\put(1097.0,131.0){\rule[-0.200pt]{0.400pt}{26.258pt}}
\put(1099.0,131.0){\rule[-0.200pt]{0.400pt}{26.258pt}}
\put(1102.0,131.0){\rule[-0.200pt]{0.400pt}{26.499pt}}
\put(1104.0,131.0){\rule[-0.200pt]{0.400pt}{26.499pt}}
\put(1107.0,131.0){\rule[-0.200pt]{0.400pt}{26.499pt}}
\put(1109.0,131.0){\rule[-0.200pt]{0.400pt}{26.740pt}}
\put(1112.0,131.0){\rule[-0.200pt]{0.400pt}{26.740pt}}
\put(1114.0,131.0){\rule[-0.200pt]{0.400pt}{26.981pt}}
\put(1117.0,131.0){\rule[-0.200pt]{0.400pt}{26.981pt}}
\put(1119.0,131.0){\rule[-0.200pt]{0.400pt}{26.981pt}}
\put(1122.0,131.0){\rule[-0.200pt]{0.400pt}{26.981pt}}
\put(1125.0,131.0){\rule[-0.200pt]{0.400pt}{26.981pt}}
\put(1127.0,131.0){\rule[-0.200pt]{0.400pt}{27.222pt}}
\put(1130.0,131.0){\rule[-0.200pt]{0.400pt}{27.222pt}}
\put(1132.0,131.0){\rule[-0.200pt]{0.400pt}{27.222pt}}
\put(1135.0,131.0){\rule[-0.200pt]{0.400pt}{27.463pt}}
\put(1137.0,131.0){\rule[-0.200pt]{0.400pt}{27.703pt}}
\put(1140.0,131.0){\rule[-0.200pt]{0.400pt}{27.944pt}}
\put(1142.0,131.0){\rule[-0.200pt]{0.400pt}{27.944pt}}
\put(1145.0,131.0){\rule[-0.200pt]{0.400pt}{27.944pt}}
\put(1147.0,131.0){\rule[-0.200pt]{0.400pt}{28.185pt}}
\put(1150.0,131.0){\rule[-0.200pt]{0.400pt}{28.426pt}}
\put(1152.0,131.0){\rule[-0.200pt]{0.400pt}{28.426pt}}
\put(1155.0,131.0){\rule[-0.200pt]{0.400pt}{28.667pt}}
\put(1158.0,131.0){\rule[-0.200pt]{0.400pt}{28.667pt}}
\put(1160.0,131.0){\rule[-0.200pt]{0.400pt}{28.908pt}}
\put(1163.0,131.0){\rule[-0.200pt]{0.400pt}{29.149pt}}
\put(1165.0,131.0){\rule[-0.200pt]{0.400pt}{29.631pt}}
\put(1168.0,131.0){\rule[-0.200pt]{0.400pt}{29.631pt}}
\put(1170.0,131.0){\rule[-0.200pt]{0.400pt}{29.872pt}}
\put(1173.0,131.0){\rule[-0.200pt]{0.400pt}{29.872pt}}
\put(1175.0,131.0){\rule[-0.200pt]{0.400pt}{30.112pt}}
\put(1178.0,131.0){\rule[-0.200pt]{0.400pt}{30.112pt}}
\put(1180.0,131.0){\rule[-0.200pt]{0.400pt}{30.594pt}}
\put(1183.0,131.0){\rule[-0.200pt]{0.400pt}{31.076pt}}
\put(1185.0,131.0){\rule[-0.200pt]{0.400pt}{31.317pt}}
\put(1188.0,131.0){\rule[-0.200pt]{0.400pt}{31.558pt}}
\put(1190.0,131.0){\rule[-0.200pt]{0.400pt}{31.558pt}}
\put(1193.0,131.0){\rule[-0.200pt]{0.400pt}{31.558pt}}
\put(1196.0,131.0){\rule[-0.200pt]{0.400pt}{31.799pt}}
\put(1198.0,131.0){\rule[-0.200pt]{0.400pt}{31.799pt}}
\put(1201.0,131.0){\rule[-0.200pt]{0.400pt}{32.762pt}}
\put(1203.0,131.0){\rule[-0.200pt]{0.400pt}{33.003pt}}
\put(1206.0,131.0){\rule[-0.200pt]{0.400pt}{33.244pt}}
\put(1208.0,131.0){\rule[-0.200pt]{0.400pt}{33.485pt}}
\put(1211.0,131.0){\rule[-0.200pt]{0.400pt}{33.726pt}}
\put(1213.0,131.0){\rule[-0.200pt]{0.400pt}{33.726pt}}
\put(1216.0,131.0){\rule[-0.200pt]{0.400pt}{33.967pt}}
\put(1218.0,131.0){\rule[-0.200pt]{0.400pt}{33.967pt}}
\put(1221.0,131.0){\rule[-0.200pt]{0.400pt}{33.967pt}}
\put(1223.0,131.0){\rule[-0.200pt]{0.400pt}{34.690pt}}
\put(1226.0,131.0){\rule[-0.200pt]{0.400pt}{35.412pt}}
\put(1229.0,131.0){\rule[-0.200pt]{0.400pt}{35.412pt}}
\put(1231.0,131.0){\rule[-0.200pt]{0.400pt}{35.412pt}}
\put(1234.0,131.0){\rule[-0.200pt]{0.400pt}{36.376pt}}
\put(1236.0,131.0){\rule[-0.200pt]{0.400pt}{37.339pt}}
\put(1239.0,131.0){\rule[-0.200pt]{0.400pt}{37.821pt}}
\put(1241.0,131.0){\rule[-0.200pt]{0.400pt}{38.785pt}}
\put(1244.0,131.0){\rule[-0.200pt]{0.400pt}{39.026pt}}
\put(1246.0,131.0){\rule[-0.200pt]{0.400pt}{39.267pt}}
\put(1249.0,131.0){\rule[-0.200pt]{0.400pt}{39.748pt}}
\put(1251.0,131.0){\rule[-0.200pt]{0.400pt}{40.230pt}}
\put(1254.0,131.0){\rule[-0.200pt]{0.400pt}{40.230pt}}
\put(1256.0,131.0){\rule[-0.200pt]{0.400pt}{42.398pt}}
\put(1259.0,131.0){\rule[-0.200pt]{0.400pt}{43.844pt}}
\put(1261.0,131.0){\rule[-0.200pt]{0.400pt}{44.807pt}}
\put(1264.0,131.0){\rule[-0.200pt]{0.400pt}{45.771pt}}
\put(1267.0,131.0){\rule[-0.200pt]{0.400pt}{46.012pt}}
\put(1269.0,131.0){\rule[-0.200pt]{0.400pt}{46.253pt}}
\put(1272.0,131.0){\rule[-0.200pt]{0.400pt}{46.735pt}}
\put(1274.0,131.0){\rule[-0.200pt]{0.400pt}{47.939pt}}
\put(1277.0,131.0){\rule[-0.200pt]{0.400pt}{50.830pt}}
\put(1279.0,131.0){\rule[-0.200pt]{0.400pt}{51.312pt}}
\put(1282.0,131.0){\rule[-0.200pt]{0.400pt}{52.516pt}}
\put(1284.0,131.0){\rule[-0.200pt]{0.400pt}{56.371pt}}
\put(1287.0,131.0){\rule[-0.200pt]{0.400pt}{58.780pt}}
\put(1289.0,131.0){\rule[-0.200pt]{0.400pt}{59.020pt}}
\put(1292.0,131.0){\rule[-0.200pt]{0.400pt}{59.261pt}}
\put(1294.0,131.0){\rule[-0.200pt]{0.400pt}{59.984pt}}
\put(1297.0,131.0){\rule[-0.200pt]{0.400pt}{61.189pt}}
\put(1300.0,131.0){\rule[-0.200pt]{0.400pt}{62.875pt}}
\put(1302.0,131.0){\rule[-0.200pt]{0.400pt}{68.897pt}}
\put(1305.0,131.0){\rule[-0.200pt]{0.400pt}{69.861pt}}
\put(1307.0,131.0){\rule[-0.200pt]{0.400pt}{80.461pt}}
\put(1310.0,131.0){\rule[-0.200pt]{0.400pt}{83.833pt}}
\put(1312.0,131.0){\rule[-0.200pt]{0.400pt}{85.038pt}}
\put(1315.0,131.0){\rule[-0.200pt]{0.400pt}{117.800pt}}
\put(1317.0,131.0){\rule[-0.200pt]{0.400pt}{129.845pt}}
\put(1320.0,131.0){\rule[-0.200pt]{0.400pt}{159.476pt}}
\put(171.0,131.0){\rule[-0.200pt]{0.400pt}{175.375pt}}
\put(171.0,131.0){\rule[-0.200pt]{305.461pt}{0.400pt}}
\put(1439.0,131.0){\rule[-0.200pt]{0.400pt}{175.375pt}}
\put(171.0,859.0){\rule[-0.200pt]{305.461pt}{0.400pt}}
\end{picture}
}
        \vspace{1cm}
        \resizebox{.9\linewidth}{!}{% GNUPLOT: LaTeX picture
\setlength{\unitlength}{0.240900pt}
\ifx\plotpoint\undefined\newsavebox{\plotpoint}\fi
\sbox{\plotpoint}{\rule[-0.200pt]{0.400pt}{0.400pt}}%
\begin{picture}(1500,900)(0,0)
\sbox{\plotpoint}{\rule[-0.200pt]{0.400pt}{0.400pt}}%
\put(171.0,131.0){\rule[-0.200pt]{4.818pt}{0.400pt}}
\put(151,131){\makebox(0,0)[r]{ 0}}
\put(1419.0,131.0){\rule[-0.200pt]{4.818pt}{0.400pt}}
\put(171.0,212.0){\rule[-0.200pt]{4.818pt}{0.400pt}}
\put(151,212){\makebox(0,0)[r]{ 0.5}}
\put(1419.0,212.0){\rule[-0.200pt]{4.818pt}{0.400pt}}
\put(171.0,293.0){\rule[-0.200pt]{4.818pt}{0.400pt}}
\put(151,293){\makebox(0,0)[r]{ 1}}
\put(1419.0,293.0){\rule[-0.200pt]{4.818pt}{0.400pt}}
\put(171.0,374.0){\rule[-0.200pt]{4.818pt}{0.400pt}}
\put(151,374){\makebox(0,0)[r]{ 1.5}}
\put(1419.0,374.0){\rule[-0.200pt]{4.818pt}{0.400pt}}
\put(171.0,455.0){\rule[-0.200pt]{4.818pt}{0.400pt}}
\put(151,455){\makebox(0,0)[r]{ 2}}
\put(1419.0,455.0){\rule[-0.200pt]{4.818pt}{0.400pt}}
\put(171.0,535.0){\rule[-0.200pt]{4.818pt}{0.400pt}}
\put(151,535){\makebox(0,0)[r]{ 2.5}}
\put(1419.0,535.0){\rule[-0.200pt]{4.818pt}{0.400pt}}
\put(171.0,616.0){\rule[-0.200pt]{4.818pt}{0.400pt}}
\put(151,616){\makebox(0,0)[r]{ 3}}
\put(1419.0,616.0){\rule[-0.200pt]{4.818pt}{0.400pt}}
\put(171.0,697.0){\rule[-0.200pt]{4.818pt}{0.400pt}}
\put(151,697){\makebox(0,0)[r]{ 3.5}}
\put(1419.0,697.0){\rule[-0.200pt]{4.818pt}{0.400pt}}
\put(171.0,778.0){\rule[-0.200pt]{4.818pt}{0.400pt}}
\put(151,778){\makebox(0,0)[r]{ 4}}
\put(1419.0,778.0){\rule[-0.200pt]{4.818pt}{0.400pt}}
\put(171.0,859.0){\rule[-0.200pt]{4.818pt}{0.400pt}}
\put(151,859){\makebox(0,0)[r]{ 4.5}}
\put(1419.0,859.0){\rule[-0.200pt]{4.818pt}{0.400pt}}
\put(171.0,131.0){\rule[-0.200pt]{0.400pt}{4.818pt}}
\put(171,90){\makebox(0,0){ 0}}
\put(171.0,839.0){\rule[-0.200pt]{0.400pt}{4.818pt}}
\put(382.0,131.0){\rule[-0.200pt]{0.400pt}{4.818pt}}
\put(382,90){\makebox(0,0){ 5}}
\put(382.0,839.0){\rule[-0.200pt]{0.400pt}{4.818pt}}
\put(594.0,131.0){\rule[-0.200pt]{0.400pt}{4.818pt}}
\put(594,90){\makebox(0,0){ 10}}
\put(594.0,839.0){\rule[-0.200pt]{0.400pt}{4.818pt}}
\put(805.0,131.0){\rule[-0.200pt]{0.400pt}{4.818pt}}
\put(805,90){\makebox(0,0){ 15}}
\put(805.0,839.0){\rule[-0.200pt]{0.400pt}{4.818pt}}
\put(1016.0,131.0){\rule[-0.200pt]{0.400pt}{4.818pt}}
\put(1016,90){\makebox(0,0){ 20}}
\put(1016.0,839.0){\rule[-0.200pt]{0.400pt}{4.818pt}}
\put(1228.0,131.0){\rule[-0.200pt]{0.400pt}{4.818pt}}
\put(1228,90){\makebox(0,0){ 25}}
\put(1228.0,839.0){\rule[-0.200pt]{0.400pt}{4.818pt}}
\put(1439.0,131.0){\rule[-0.200pt]{0.400pt}{4.818pt}}
\put(1439,90){\makebox(0,0){ 30}}
\put(1439.0,839.0){\rule[-0.200pt]{0.400pt}{4.818pt}}
\put(171.0,131.0){\rule[-0.200pt]{0.400pt}{175.375pt}}
\put(171.0,131.0){\rule[-0.200pt]{305.461pt}{0.400pt}}
\put(1439.0,131.0){\rule[-0.200pt]{0.400pt}{175.375pt}}
\put(171.0,859.0){\rule[-0.200pt]{305.461pt}{0.400pt}}
\put(30,495){\makebox(0,0){Speedup}}
\put(805,29){\makebox(0,0){Files of Polybench}}
\put(171.0,131.0){\rule[-0.200pt]{0.400pt}{8.431pt}}
\put(213.0,131.0){\rule[-0.200pt]{0.400pt}{11.804pt}}
\put(256.0,131.0){\rule[-0.200pt]{0.400pt}{12.527pt}}
\put(298.0,131.0){\rule[-0.200pt]{0.400pt}{13.490pt}}
\put(340.0,131.0){\rule[-0.200pt]{0.400pt}{14.936pt}}
\put(382.0,131.0){\rule[-0.200pt]{0.400pt}{16.140pt}}
\put(425.0,131.0){\rule[-0.200pt]{0.400pt}{18.067pt}}
\put(467.0,131.0){\rule[-0.200pt]{0.400pt}{21.681pt}}
\put(509.0,131.0){\rule[-0.200pt]{0.400pt}{22.404pt}}
\put(551.0,131.0){\rule[-0.200pt]{0.400pt}{25.054pt}}
\put(594.0,131.0){\rule[-0.200pt]{0.400pt}{25.054pt}}
\put(636.0,131.0){\rule[-0.200pt]{0.400pt}{25.054pt}}
\put(678.0,131.0){\rule[-0.200pt]{0.400pt}{25.294pt}}
\put(720.0,131.0){\rule[-0.200pt]{0.400pt}{26.258pt}}
\put(763.0,131.0){\rule[-0.200pt]{0.400pt}{27.222pt}}
\put(805.0,131.0){\rule[-0.200pt]{0.400pt}{34.208pt}}
\put(847.0,131.0){\rule[-0.200pt]{0.400pt}{35.412pt}}
\put(890.0,131.0){\rule[-0.200pt]{0.400pt}{36.376pt}}
\put(932.0,131.0){\rule[-0.200pt]{0.400pt}{39.748pt}}
\put(974.0,131.0){\rule[-0.200pt]{0.400pt}{40.712pt}}
\put(1016.0,131.0){\rule[-0.200pt]{0.400pt}{41.435pt}}
\put(1059.0,131.0){\rule[-0.200pt]{0.400pt}{43.121pt}}
\put(1101.0,131.0){\rule[-0.200pt]{0.400pt}{44.566pt}}
\put(1143.0,131.0){\rule[-0.200pt]{0.400pt}{45.530pt}}
\put(1185.0,131.0){\rule[-0.200pt]{0.400pt}{48.421pt}}
\put(1228.0,131.0){\rule[-0.200pt]{0.400pt}{50.348pt}}
\put(1270.0,131.0){\rule[-0.200pt]{0.400pt}{50.830pt}}
\put(1312.0,131.0){\rule[-0.200pt]{0.400pt}{55.407pt}}
\put(1354.0,131.0){\rule[-0.200pt]{0.400pt}{78.292pt}}
\put(1397.0,131.0){\rule[-0.200pt]{0.400pt}{89.615pt}}
\put(1439.0,131.0){\rule[-0.200pt]{0.400pt}{166.944pt}}
\put(171.0,131.0){\rule[-0.200pt]{0.400pt}{175.375pt}}
\put(171.0,131.0){\rule[-0.200pt]{305.461pt}{0.400pt}}
\put(1439.0,131.0){\rule[-0.200pt]{0.400pt}{175.375pt}}
\put(171.0,859.0){\rule[-0.200pt]{305.461pt}{0.400pt}}
\end{picture}
}
      \end{center}
  \end{columns}
\end{frame}

\frame{\frametitle{Conclusion and Future work}
  Conclusion
  \begin{itemize}
  \item New faster algorithm for SCoP detection
  \item Enable polyhedral optimization in industrial compilers
  \end{itemize}

  \vspace{1cm}
  
  Future Work
  \begin{itemize}
  \item SCoP detection to drive polyhedral optimization \\
    (avoid maximal SCoPs)
  \item Use profile data to guide and select polyhedral transforms
  \end{itemize}
}


\end{document}
