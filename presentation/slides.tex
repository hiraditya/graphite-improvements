\documentclass{beamer}
\begin{document}
\def \SCoP {SCoP}
\def \GCC {GCC}
\def \LLVM {LLVM}
\def \SESE {SESE}
\def \CFG {CFG}
\def \SSA {SSA}
\def \scev {scev}

\title{\SCoP{} Detection: A Fast Algorithm for Industrial Compilers}
\author{Sebastian Pop and Aditya Kumar}
\institute{SARC: Samsung Austin R\&D Center}
\date{Jan 19, 2016}
%\date{\today}

\frame{\titlepage}

\frame{\frametitle{SCoP Detection (Introduction) }
  \begin{itemize}
    \item Single Entry Single Exit (SESE) regions as Static Controlled Parts (SCoP).
    \item Loop Tree data structure.
  \end{itemize}
}

\frame{\frametitle{SCoP Detection (Other implementations) }
  \begin{itemize}
    \item Previous graphite SCoP detection based on CFG and DOM.
    \item Polly's SCoP detection based on analysis of SESE regions.
  \end{itemize}
}

\begin{frame}[fragile]{SCoP Detection (New Algorithm) }
\begin{verbatim}
// Recurse on the loop.inner.
sese build_scop_depth (sese s1, loop l):
  s1 = build_scop_depth (s1, l.inner)
  sese s2 = merge_sese (s1, get_sese (l))
  if (s2 is an invalid scop)
    {
      // s1 might be a valid scop, so return it
      // and start analyzing from the adjacent loop.
      build_scop_depth (invalid_sese, l.next)
      return s1
    }
  if (l is an invalid scop in s2)
    return build_scop_depth (invalid_sese, l.next)
  return build_scop_breadth (s2, l)

\end{verbatim}
\end{frame}

\begin{frame}[fragile]{SCoP Detection (New Algorithm) }
\begin{verbatim}
// Recurse on loop.next.
sese build_scop_breadth (sese s1, loop l):
  sese s2 = build_scop_depth (invalid_sese, l.next)
  if (s2 is an invalid scop)
    {
      if (s1 is a valid scop)
        add_scop (s1)
      return s1
    }
  sese combined = merge_sese (s1, s2)
  if (combined is a valid scop)
    s1 = combined
  else
    add_scop (s2)
  if (s1 is a valid scop)
    add_scop (s1)
  return s1
\end{verbatim}
\end{frame}

\frame{\frametitle{Outline of the presentation}
  \begin{itemize}
    \item Introduction
    \item Details of algorithm
    \item Experimental results
    \item Conclusion and future work
  \end{itemize}
}

\frame{\frametitle{Conclusion and Future work}
  Conclusion
  \begin{itemize}
    \item New algorithm for SCoP detection.
    \item Comparative analysis to show compile time improvement.
  \end{itemize}

  Future Work
  \begin{itemize}
    \item SCoP detection should drive polyhedral optimization. 
    \item Using profile information to tune the SCoP detection.
  \end{itemize}
}


\frame{\frametitle{Experimental results}
  \begin{itemize}
    \item Improvements on polybench.
  \end{itemize}
}

\end{document}
