\documentclass{beamer}
\usepackage{graphicx}
\usepackage{amssymb}
\usepackage{graphviz}
\usepackage{tikz}
%\usepackage{auto-pst-pdf}
\usepackage{dot2texi}
\usepackage{listings}

\pgfdeclarelayer{background}
\pgfdeclarelayer{foreground}
\pgfsetlayers{background,main,foreground}

\begin{document}
\def \SCoP {SCoP}
\def \GCC {GCC}
\def \LLVM {LLVM}
\def \SESE {SESE}
\def \CFG {CFG}
\def \SSA {SSA}
\def \scev {scev}

\title{\SCoP{} Detection: A Fast Algorithm for Industrial Compilers}
\author{Sebastian Pop and Aditya Kumar}
\institute{SARC: Samsung Austin R\&D Center}
\date{Jan 19, 2016}
%\date{\today}

\definecolor{myblue}{rgb}{0.0, 0.0, 0.5}
\definecolor{myred}{rgb}{0.5, 0.0, 0.0}
\definecolor{mygreen}{rgb}{0.0, 0.5, 0.0}
\lstset{language=C++,
  basicstyle=\ttfamily,
  keywordstyle=\color{myblue}\ttfamily,
  stringstyle=\color{myred}\ttfamily,
  commentstyle=\color{mygreen}\ttfamily,
  morecomment=[l][\color{magenta}]{\#}
}
\addtobeamertemplate{navigation symbols}{}{%
    \usebeamerfont{footline}%
    \usebeamercolor[fg]{footline}%
    \hspace{1em}%
    \insertframenumber/\inserttotalframenumber
}

\frame{\titlepage}

\frame{\frametitle{What is a SCoP?}
  Regions of code that can be represented in the Polyhedral Model. \\
  We use isl (the integer set library) for scheduling and transforms.
  \begin{itemize}
  \item SCoPs = Static Control Parts
    \pause
  \item ACLs =
    \pause
    Affine Control Loops
  \item PWACs =
    \pause
    Parts With Affine Control, rhymes with quacks :-)
  \end{itemize}
}

\begin{frame}[fragile]{Step 1: accept natural loops}
  \begin{columns}[T,onlytextwidth] % align columns
    \column{.2\textwidth}
    \begin{block}{\small Natural loop}
      \begin{tikzpicture}[scale=0.4]
        \begin{dot2tex}[dot,tikz,codeonly,styleonly]
          digraph G {
            {rank = same; a; x;}
            e -> a -> x;
            a -> b -> a [fontcolor=red];
          }
        \end{dot2tex}
      \end{tikzpicture}

      {\small \color{myblue} maybe SCoP}
    \end{block}
    \pause
    \column{.2\textwidth}
    \begin{block}{\small Nested loops}
      \begin{tikzpicture}[scale=0.4]
        \begin{dot2tex}[dot,tikz,codeonly,styleonly]
          digraph G {
            {rank = same; a; x;}
            e -> a -> x;
            a -> b -> d -> a;
            b -> c -> b;
          }
        \end{dot2tex}
      \end{tikzpicture}

      {\small \color{myblue} maybe SCoP}
    \end{block}
    \pause
    \column{.2\textwidth}
    \begin{block}{\small Irreducible}
      \begin{tikzpicture}[scale=0.4]
        \begin{dot2tex}[dot,tikz,codeonly,styleonly]
          digraph G {
            {rank = same; a; x;}
            e -> a -> x;
            a -> b -> a;
            a -> c -> a;
          }
        \end{dot2tex}
      \end{tikzpicture}

      {\small \color{myred} not a SCoP: ambiguous iteration order}
    \end{block}
    \pause
    \column{.2\textwidth}
    \begin{block}{\small Irreducible}
      \begin{tikzpicture}[scale=0.4]
        \begin{dot2tex}[dot,tikz,codeonly,styleonly]
          digraph G {
            {rank = same; a; x;}
            e -> a -> x;
            a -> b -> a;
            b -> c -> b;
          }
        \end{dot2tex}
      \end{tikzpicture}

      {\small \color{myred} not a SCoP: ambiguous iteration order}
    \end{block}
  \end{columns}
\end{frame}

\begin{frame}[fragile]{Natural Loop Tree}
  \begin{columns}[T,onlytextwidth] % align columns
    \column{0.7\textwidth}
    \begin{lstlisting}
int foo(int N)
{
 int i, j, k;
 for(i=0; i<N; ++i){//Loop1
  stmt1;
  for (j=0; j<N; ++j)//Loop2
   stmt2;
  for (k=0; k<N; ++k)//Loop3
   stmt3;
 }
}
    \end{lstlisting}
    \pause
    \column{.3\textwidth}
    \begin{tikzpicture}[scale=0.7]
      % set node style
      \tikzstyle{n} = [draw,shape=circle,minimum size=2em,
        inner sep=0pt,fill=red!20]
      \begin{dot2tex}[dot,tikz,codeonly,styleonly,options=-s -tmath]
        digraph G  {
          node [style="n"];
          {rank = same; Loop_2; Loop_3;}
          Function -> Loop_1 [label=" inner"];
          Loop_1 -> Loop_2 [label=" inner", orientation=90];
          Loop_2 -> Loop_3 [label="next"];
        }
      \end{dot2tex}
    \end{tikzpicture}
  \end{columns}
\end{frame}

%% \begin{frame}[fragile]{Step 2: accept structured control flow}
%%   \begin{columns}[T,onlytextwidth] % align columns
%%     \column{.3\textwidth}
%%     \begin{block}{\small simple condition}
%%       \begin{tikzpicture}[scale=0.4]
%%         \begin{dot2tex}[dot,tikz,codeonly,styleonly]
%%           digraph G {
%%             e -> c -> a ->x;
%%             c -> b -> x;
%%           }
%%         \end{dot2tex}
%%       \end{tikzpicture}

%%       {\small \color{myblue} maybe SCoP}
%%     \end{block}
%%     \pause
%%     \column{.3\textwidth}
%%     \begin{block}{\small nested conditions}
%%       \begin{tikzpicture}[scale=0.4]
%%         \begin{dot2tex}[dot,tikz,codeonly,styleonly]
%%           digraph G {
%%             e -> c -> b -> x;
%%             c -> a -> f -> x;
%%             a -> d -> x;
%%           }
%%         \end{dot2tex}
%%       \end{tikzpicture}

%%       {\small \color{myblue} maybe SCoP}
%%     \end{block}
%%     \pause
%%     \column{.3\textwidth}
%%     \begin{block}{\small unstructured}
%%       \begin{tikzpicture}[scale=0.4]
%%         \begin{dot2tex}[dot,tikz,codeonly,styleonly]
%%           digraph G {
%%             e -> c -> b -> d -> x;
%%             c -> a -> f -> x;
%%             a -> d;
%%           }
%%         \end{dot2tex}
%%       \end{tikzpicture}

%%       {\small \color{myred} not a SCoP: control dependences are hard}
%%     \end{block}
%%   \end{columns}
%% \end{frame}

\begin{frame}[fragile]{Step 2: check for side-effects}
  \begin{itemize}
  \item function calls
  \item inline assembly
  \item volatile operations
  \end{itemize}
\end{frame}

\begin{frame}[fragile]{Step 3: affine scalar evolutions}
  \begin{columns}[T,onlytextwidth] % align columns
    \column{.6\textwidth}
    \begin{block}{\small Linear}
      \begin{lstlisting}
i0 = phi_l1(0, i1)
// i0={0,+,1}_l1
i1 = i0 + 1
// i1={1,+,1}_l1
      \end{lstlisting}

      {\small \color{myblue} maybe SCoP}
    \end{block}
    \pause
    \begin{block}{\small Non-linear} 
      \begin{lstlisting}
j2 = phi_l1(3, j3)
j3 = j2 + i1
// j2={3,+,{1,+,1}_l1}_l1
      \end{lstlisting}
      {\small \color{myred} not an ACL: polynomial of degree 2}
    \end{block}
    \pause
    \column{.4\textwidth}
    \begin{block}{\small Non-linear} 
      \begin{lstlisting}
k4 = phi_l2(4, k5)
k5 = k4 * 2
// k4={4,*,2}_l2
      \end{lstlisting}
      {\small \color{myred} not an ACL: exponential}
    \end{block}

    \vspace{.5cm}
    \pause
    \begin{block}{\small analyzed expressions}
      \begin{itemize}
      \item branch conditions
      \item memory accesses
      \end{itemize}
    \end{block}
  \end{columns}
\end{frame}

\begin{frame}[fragile]{Step 4: delinearize memory access functions}
  \begin{columns}[T,onlytextwidth] % align columns
    \column{.6\textwidth}
    \begin{block}{\small Linear access functions}
      \begin{lstlisting}
A[100*i + 400*j]
B[i][j]
      \end{lstlisting}
      {\small \color{myblue} can represent in isl}
    \end{block}

    \vspace{.7cm}
    \pause
    \begin{block}{\small Non-linear access functions} 
      \begin{lstlisting}
C[i*i]
D[4*N*M*i + 4*M*j + 4*k]
E[4*i*N + 4*j]
      \end{lstlisting}

      {\small \color{myred} cannot represent in isl}
    \end{block}
    \pause
    \column{.4\textwidth}
    \begin{block}{\small delinearization}
      \begin{itemize}
      \item recognize array multi-dimensions
      \item compute linear access functions
      \end{itemize}
    \end{block}
    \pause
    \begin{block}{\small delinearized access functions}
      \begin{lstlisting}
int D[][N][M];
D[i][j][k]

int E[][N];
E[i][j]
      \end{lstlisting}

      {\small \color{myblue} can represent in isl}
    \end{block}
  \end{columns}
\end{frame}

\begin{frame}[fragile]{Overall picture: SCoP detection}
  \begin{columns}[T,onlytextwidth] % align columns
    \column{.4\textwidth}
    \begin{tikzpicture}[scale=0.4]
      \tikzstyle{n} = [draw,shape=rectangle]
      \begin{dot2tex}[dot,tikz,codeonly,styleonly]
        digraph G {
          node [style="n"];
          "Natural loops"
          -> "no side-effects?"
          -> "affine branch conditions?"
          -> "affine memory accesses?"
          -> "SCoP";
        }
      \end{dot2tex}
    \end{tikzpicture}
    \pause
    \column{.5\textwidth}
    Required analyses:
    \begin{itemize}
    \item natural loops tree
    \item (post-)dominators tree
    \item alias analysis
    \item scalar evolution analysis
    \end{itemize}
  \end{columns}
\end{frame}

\frame{\frametitle{Detecting SCoPs by induction on Natural Loops Tree}
  \begin{itemize}
  \item Start with a loop in the natural loops tree \\
    rather than the root of the CFG
    \vspace{1cm}
    \pause
  \item Focus on structure of natural loops \\
    before the validity of each statement
  \end{itemize}
}

\begin{frame}[fragile]{Example: Induction on Natural Loops Tree}
  \begin{tikzpicture}[scale=0.8]
    % set node style
    \tikzstyle{n} = [draw,shape=circle,minimum size=2em,
      inner sep=0pt,fill=red!20]
    \begin{dot2tex}[dot,tikz,codeonly,styleonly,options=-s -tmath]
      digraph G  {
        node [style="n"];
        {rank = same; Loop_2; Loop_3;}
        Function -> Loop_1 [label="inner"];
        Loop_1 -> Loop_2 [label="inner", orientation=90];
        Loop_2 -> Loop_3 [label="next"];
      }
    \end{dot2tex}
    \begin{pgfonlayer}{background}
      \pause
      \draw[rounded corners=2em,line width=3em,blue!30,cap=round]
      (Loop_2.center) -- (Loop_2.east);
      \pause
      \draw[rounded corners=2em,line width=3em,blue!30,cap=round]
      (Loop_3.west) -- (Loop_3.center);
      \pause
      \draw[rounded corners=2em,line width=3em,blue!30,cap=round]
      (Loop_2.center) -- (Loop_3.center);
      \pause
      \draw[rounded corners=2em,line width=3em,blue!30,cap=round]
      (Loop_1.center) -- (Loop_2.west) -- (Loop_3.east) -- (Loop_1.center);
      \fill[rounded corners=2em,line width=3em,blue!30,cap=round]
      (Loop_1.center) -- (Loop_2.center) -- (Loop_3.center);
    \end{pgfonlayer}
  \end{tikzpicture}
\end{frame}

\frame{\frametitle{Other implementations of SCoP Detection}
  \begin{itemize}
  \item Previous graphite SCoP detection based on CFG and DOM \\
    (misses the structure of loops)

    \vspace{1cm}
    \pause

  \item Polly's SCoP detection based on structure of SESE regions \\
    (full function body analysis even without interesting loops)

    \vspace{1cm}
    \pause

  \item Pet, Rose, other source-to-source compilers: SCoP detection \\
    based on the AST of a specific programming language
  \end{itemize}
}

\begin{frame}[fragile]{Experimental Results}
  \begin{columns}[T,onlytextwidth] % align columns
    \column{.4\textwidth}
    \begin{block}{\small Compilation time overhead}
      \resizebox{\linewidth}{!}{
        \begin{tabular}{|c|c|c|}
          \hline
          Benchmark  & Old \% & New \%  \\
          \hline
          Polybench  & $1.4$  & $1.9$   \\
          Tramp3d-v4 & $7.0$  & $0.3$   \\
          \GCC{} 6.0 & $0.24$ & $0.01$  \\
          \hline
        \end{tabular}
      }
    \end{block}

    \vspace{2cm} 
    \begin{block}{\small SCoP Metrics on Polybench}
      \resizebox{\linewidth}{!}{
        \begin{tabular}{|c|c|c|c|}
          \hline
          SCoP Metric   & Old  & New  & Polly  \\
          \hline
          Loops/\SCoP{} & 2.59 & 6.09 & 5.17   \\
          \hline
        \end{tabular}
      }
    \end{block}

    \column{.6\textwidth}
      \begin{center}
        \resizebox{.9\linewidth}{!}{% GNUPLOT: LaTeX picture
\setlength{\unitlength}{0.240900pt}
\ifx\plotpoint\undefined\newsavebox{\plotpoint}\fi
\sbox{\plotpoint}{\rule[-0.200pt]{0.400pt}{0.400pt}}%
\begin{picture}(1500,900)(0,0)
\sbox{\plotpoint}{\rule[-0.200pt]{0.400pt}{0.400pt}}%
\put(231.0,131.0){\rule[-0.200pt]{4.818pt}{0.400pt}}
\put(211,131){\makebox(0,0)[r]{ 1}}
\put(1419.0,131.0){\rule[-0.200pt]{4.818pt}{0.400pt}}
\put(231.0,168.0){\rule[-0.200pt]{2.409pt}{0.400pt}}
\put(1429.0,168.0){\rule[-0.200pt]{2.409pt}{0.400pt}}
\put(231.0,216.0){\rule[-0.200pt]{2.409pt}{0.400pt}}
\put(1429.0,216.0){\rule[-0.200pt]{2.409pt}{0.400pt}}
\put(231.0,241.0){\rule[-0.200pt]{2.409pt}{0.400pt}}
\put(1429.0,241.0){\rule[-0.200pt]{2.409pt}{0.400pt}}
\put(231.0,252.0){\rule[-0.200pt]{4.818pt}{0.400pt}}
\put(211,252){\makebox(0,0)[r]{ 10}}
\put(1419.0,252.0){\rule[-0.200pt]{4.818pt}{0.400pt}}
\put(231.0,289.0){\rule[-0.200pt]{2.409pt}{0.400pt}}
\put(1429.0,289.0){\rule[-0.200pt]{2.409pt}{0.400pt}}
\put(231.0,337.0){\rule[-0.200pt]{2.409pt}{0.400pt}}
\put(1429.0,337.0){\rule[-0.200pt]{2.409pt}{0.400pt}}
\put(231.0,362.0){\rule[-0.200pt]{2.409pt}{0.400pt}}
\put(1429.0,362.0){\rule[-0.200pt]{2.409pt}{0.400pt}}
\put(231.0,374.0){\rule[-0.200pt]{4.818pt}{0.400pt}}
\put(211,374){\makebox(0,0)[r]{ 100}}
\put(1419.0,374.0){\rule[-0.200pt]{4.818pt}{0.400pt}}
\put(231.0,410.0){\rule[-0.200pt]{2.409pt}{0.400pt}}
\put(1429.0,410.0){\rule[-0.200pt]{2.409pt}{0.400pt}}
\put(231.0,458.0){\rule[-0.200pt]{2.409pt}{0.400pt}}
\put(1429.0,458.0){\rule[-0.200pt]{2.409pt}{0.400pt}}
\put(231.0,483.0){\rule[-0.200pt]{2.409pt}{0.400pt}}
\put(1429.0,483.0){\rule[-0.200pt]{2.409pt}{0.400pt}}
\put(231.0,495.0){\rule[-0.200pt]{4.818pt}{0.400pt}}
\put(211,495){\makebox(0,0)[r]{ 1000}}
\put(1419.0,495.0){\rule[-0.200pt]{4.818pt}{0.400pt}}
\put(231.0,532.0){\rule[-0.200pt]{2.409pt}{0.400pt}}
\put(1429.0,532.0){\rule[-0.200pt]{2.409pt}{0.400pt}}
\put(231.0,580.0){\rule[-0.200pt]{2.409pt}{0.400pt}}
\put(1429.0,580.0){\rule[-0.200pt]{2.409pt}{0.400pt}}
\put(231.0,605.0){\rule[-0.200pt]{2.409pt}{0.400pt}}
\put(1429.0,605.0){\rule[-0.200pt]{2.409pt}{0.400pt}}
\put(231.0,616.0){\rule[-0.200pt]{4.818pt}{0.400pt}}
\put(211,616){\makebox(0,0)[r]{ 10000}}
\put(1419.0,616.0){\rule[-0.200pt]{4.818pt}{0.400pt}}
\put(231.0,653.0){\rule[-0.200pt]{2.409pt}{0.400pt}}
\put(1429.0,653.0){\rule[-0.200pt]{2.409pt}{0.400pt}}
\put(231.0,701.0){\rule[-0.200pt]{2.409pt}{0.400pt}}
\put(1429.0,701.0){\rule[-0.200pt]{2.409pt}{0.400pt}}
\put(231.0,726.0){\rule[-0.200pt]{2.409pt}{0.400pt}}
\put(1429.0,726.0){\rule[-0.200pt]{2.409pt}{0.400pt}}
\put(231.0,738.0){\rule[-0.200pt]{4.818pt}{0.400pt}}
\put(211,738){\makebox(0,0)[r]{ 100000}}
\put(1419.0,738.0){\rule[-0.200pt]{4.818pt}{0.400pt}}
\put(231.0,774.0){\rule[-0.200pt]{2.409pt}{0.400pt}}
\put(1429.0,774.0){\rule[-0.200pt]{2.409pt}{0.400pt}}
\put(231.0,822.0){\rule[-0.200pt]{2.409pt}{0.400pt}}
\put(1429.0,822.0){\rule[-0.200pt]{2.409pt}{0.400pt}}
\put(231.0,847.0){\rule[-0.200pt]{2.409pt}{0.400pt}}
\put(1429.0,847.0){\rule[-0.200pt]{2.409pt}{0.400pt}}
\put(231.0,859.0){\rule[-0.200pt]{4.818pt}{0.400pt}}
\put(211,859){\makebox(0,0)[r]{ 1e+06}}
\put(1419.0,859.0){\rule[-0.200pt]{4.818pt}{0.400pt}}
\put(231.0,131.0){\rule[-0.200pt]{0.400pt}{4.818pt}}
\put(231,90){\makebox(0,0){ 0}}
\put(231.0,839.0){\rule[-0.200pt]{0.400pt}{4.818pt}}
\put(352.0,131.0){\rule[-0.200pt]{0.400pt}{4.818pt}}
\put(352,90){\makebox(0,0){ 50}}
\put(352.0,839.0){\rule[-0.200pt]{0.400pt}{4.818pt}}
\put(473.0,131.0){\rule[-0.200pt]{0.400pt}{4.818pt}}
\put(473,90){\makebox(0,0){ 100}}
\put(473.0,839.0){\rule[-0.200pt]{0.400pt}{4.818pt}}
\put(593.0,131.0){\rule[-0.200pt]{0.400pt}{4.818pt}}
\put(593,90){\makebox(0,0){ 150}}
\put(593.0,839.0){\rule[-0.200pt]{0.400pt}{4.818pt}}
\put(714.0,131.0){\rule[-0.200pt]{0.400pt}{4.818pt}}
\put(714,90){\makebox(0,0){ 200}}
\put(714.0,839.0){\rule[-0.200pt]{0.400pt}{4.818pt}}
\put(835.0,131.0){\rule[-0.200pt]{0.400pt}{4.818pt}}
\put(835,90){\makebox(0,0){ 250}}
\put(835.0,839.0){\rule[-0.200pt]{0.400pt}{4.818pt}}
\put(956.0,131.0){\rule[-0.200pt]{0.400pt}{4.818pt}}
\put(956,90){\makebox(0,0){ 300}}
\put(956.0,839.0){\rule[-0.200pt]{0.400pt}{4.818pt}}
\put(1077.0,131.0){\rule[-0.200pt]{0.400pt}{4.818pt}}
\put(1077,90){\makebox(0,0){ 350}}
\put(1077.0,839.0){\rule[-0.200pt]{0.400pt}{4.818pt}}
\put(1197.0,131.0){\rule[-0.200pt]{0.400pt}{4.818pt}}
\put(1197,90){\makebox(0,0){ 400}}
\put(1197.0,839.0){\rule[-0.200pt]{0.400pt}{4.818pt}}
\put(1318.0,131.0){\rule[-0.200pt]{0.400pt}{4.818pt}}
\put(1318,90){\makebox(0,0){ 450}}
\put(1318.0,839.0){\rule[-0.200pt]{0.400pt}{4.818pt}}
\put(1439.0,131.0){\rule[-0.200pt]{0.400pt}{4.818pt}}
\put(1439,90){\makebox(0,0){ 500}}
\put(1439.0,839.0){\rule[-0.200pt]{0.400pt}{4.818pt}}
\put(231.0,131.0){\rule[-0.200pt]{0.400pt}{175.375pt}}
\put(231.0,131.0){\rule[-0.200pt]{291.007pt}{0.400pt}}
\put(1439.0,131.0){\rule[-0.200pt]{0.400pt}{175.375pt}}
\put(231.0,859.0){\rule[-0.200pt]{291.007pt}{0.400pt}}
\put(30,495){\makebox(0,0){Speedup}}
\put(835,29){\makebox(0,0){Files of GCC 6.0}}
\put(231.0,131.0){\rule[-0.200pt]{0.400pt}{5.059pt}}
\put(233.0,131.0){\rule[-0.200pt]{0.400pt}{6.745pt}}
\put(236.0,131.0){\rule[-0.200pt]{0.400pt}{7.950pt}}
\put(238.0,131.0){\rule[-0.200pt]{0.400pt}{7.950pt}}
\put(241.0,131.0){\rule[-0.200pt]{0.400pt}{13.972pt}}
\put(243.0,131.0){\rule[-0.200pt]{0.400pt}{15.418pt}}
\put(245.0,131.0){\rule[-0.200pt]{0.400pt}{15.658pt}}
\put(248.0,131.0){\rule[-0.200pt]{0.400pt}{17.104pt}}
\put(250.0,131.0){\rule[-0.200pt]{0.400pt}{17.345pt}}
\put(253.0,131.0){\rule[-0.200pt]{0.400pt}{17.345pt}}
\put(255.0,131.0){\rule[-0.200pt]{0.400pt}{18.790pt}}
\put(258.0,131.0){\rule[-0.200pt]{0.400pt}{18.790pt}}
\put(260.0,131.0){\rule[-0.200pt]{0.400pt}{18.790pt}}
\put(262.0,131.0){\rule[-0.200pt]{0.400pt}{19.513pt}}
\put(265.0,131.0){\rule[-0.200pt]{0.400pt}{20.236pt}}
\put(267.0,131.0){\rule[-0.200pt]{0.400pt}{21.440pt}}
\put(270.0,131.0){\rule[-0.200pt]{0.400pt}{21.681pt}}
\put(272.0,131.0){\rule[-0.200pt]{0.400pt}{21.681pt}}
\put(274.0,131.0){\rule[-0.200pt]{0.400pt}{21.922pt}}
\put(277.0,131.0){\rule[-0.200pt]{0.400pt}{22.163pt}}
\put(279.0,131.0){\rule[-0.200pt]{0.400pt}{22.885pt}}
\put(282.0,131.0){\rule[-0.200pt]{0.400pt}{23.126pt}}
\put(284.0,131.0){\rule[-0.200pt]{0.400pt}{23.126pt}}
\put(287.0,131.0){\rule[-0.200pt]{0.400pt}{23.367pt}}
\put(289.0,131.0){\rule[-0.200pt]{0.400pt}{23.849pt}}
\put(291.0,131.0){\rule[-0.200pt]{0.400pt}{24.331pt}}
\put(294.0,131.0){\rule[-0.200pt]{0.400pt}{24.331pt}}
\put(296.0,131.0){\rule[-0.200pt]{0.400pt}{24.572pt}}
\put(299.0,131.0){\rule[-0.200pt]{0.400pt}{24.572pt}}
\put(301.0,131.0){\rule[-0.200pt]{0.400pt}{24.813pt}}
\put(303.0,131.0){\rule[-0.200pt]{0.400pt}{25.054pt}}
\put(306.0,131.0){\rule[-0.200pt]{0.400pt}{25.294pt}}
\put(308.0,131.0){\rule[-0.200pt]{0.400pt}{25.535pt}}
\put(311.0,131.0){\rule[-0.200pt]{0.400pt}{25.535pt}}
\put(313.0,131.0){\rule[-0.200pt]{0.400pt}{25.535pt}}
\put(316.0,131.0){\rule[-0.200pt]{0.400pt}{25.776pt}}
\put(318.0,131.0){\rule[-0.200pt]{0.400pt}{25.776pt}}
\put(320.0,131.0){\rule[-0.200pt]{0.400pt}{25.776pt}}
\put(323.0,131.0){\rule[-0.200pt]{0.400pt}{26.258pt}}
\put(325.0,131.0){\rule[-0.200pt]{0.400pt}{26.258pt}}
\put(328.0,131.0){\rule[-0.200pt]{0.400pt}{26.740pt}}
\put(330.0,131.0){\rule[-0.200pt]{0.400pt}{26.981pt}}
\put(332.0,131.0){\rule[-0.200pt]{0.400pt}{27.463pt}}
\put(335.0,131.0){\rule[-0.200pt]{0.400pt}{27.703pt}}
\put(337.0,131.0){\rule[-0.200pt]{0.400pt}{27.703pt}}
\put(340.0,131.0){\rule[-0.200pt]{0.400pt}{28.908pt}}
\put(342.0,131.0){\rule[-0.200pt]{0.400pt}{28.908pt}}
\put(345.0,131.0){\rule[-0.200pt]{0.400pt}{28.908pt}}
\put(347.0,131.0){\rule[-0.200pt]{0.400pt}{29.631pt}}
\put(349.0,131.0){\rule[-0.200pt]{0.400pt}{30.112pt}}
\put(352.0,131.0){\rule[-0.200pt]{0.400pt}{30.594pt}}
\put(354.0,131.0){\rule[-0.200pt]{0.400pt}{30.835pt}}
\put(357.0,131.0){\rule[-0.200pt]{0.400pt}{30.835pt}}
\put(359.0,131.0){\rule[-0.200pt]{0.400pt}{31.076pt}}
\put(361.0,131.0){\rule[-0.200pt]{0.400pt}{31.317pt}}
\put(364.0,131.0){\rule[-0.200pt]{0.400pt}{31.799pt}}
\put(366.0,131.0){\rule[-0.200pt]{0.400pt}{32.040pt}}
\put(369.0,131.0){\rule[-0.200pt]{0.400pt}{32.040pt}}
\put(371.0,131.0){\rule[-0.200pt]{0.400pt}{32.281pt}}
\put(374.0,131.0){\rule[-0.200pt]{0.400pt}{32.281pt}}
\put(376.0,131.0){\rule[-0.200pt]{0.400pt}{32.521pt}}
\put(378.0,131.0){\rule[-0.200pt]{0.400pt}{32.521pt}}
\put(381.0,131.0){\rule[-0.200pt]{0.400pt}{32.521pt}}
\put(383.0,131.0){\rule[-0.200pt]{0.400pt}{32.762pt}}
\put(386.0,131.0){\rule[-0.200pt]{0.400pt}{32.762pt}}
\put(388.0,131.0){\rule[-0.200pt]{0.400pt}{33.003pt}}
\put(390.0,131.0){\rule[-0.200pt]{0.400pt}{33.003pt}}
\put(393.0,131.0){\rule[-0.200pt]{0.400pt}{33.003pt}}
\put(395.0,131.0){\rule[-0.200pt]{0.400pt}{33.244pt}}
\put(398.0,131.0){\rule[-0.200pt]{0.400pt}{33.485pt}}
\put(400.0,131.0){\rule[-0.200pt]{0.400pt}{33.485pt}}
\put(403.0,131.0){\rule[-0.200pt]{0.400pt}{33.485pt}}
\put(405.0,131.0){\rule[-0.200pt]{0.400pt}{33.726pt}}
\put(407.0,131.0){\rule[-0.200pt]{0.400pt}{33.726pt}}
\put(410.0,131.0){\rule[-0.200pt]{0.400pt}{33.967pt}}
\put(412.0,131.0){\rule[-0.200pt]{0.400pt}{33.967pt}}
\put(415.0,131.0){\rule[-0.200pt]{0.400pt}{33.967pt}}
\put(417.0,131.0){\rule[-0.200pt]{0.400pt}{34.208pt}}
\put(419.0,131.0){\rule[-0.200pt]{0.400pt}{34.208pt}}
\put(422.0,131.0){\rule[-0.200pt]{0.400pt}{34.208pt}}
\put(424.0,131.0){\rule[-0.200pt]{0.400pt}{34.449pt}}
\put(427.0,131.0){\rule[-0.200pt]{0.400pt}{34.449pt}}
\put(429.0,131.0){\rule[-0.200pt]{0.400pt}{34.449pt}}
\put(432.0,131.0){\rule[-0.200pt]{0.400pt}{34.449pt}}
\put(434.0,131.0){\rule[-0.200pt]{0.400pt}{34.690pt}}
\put(436.0,131.0){\rule[-0.200pt]{0.400pt}{34.930pt}}
\put(439.0,131.0){\rule[-0.200pt]{0.400pt}{34.930pt}}
\put(441.0,131.0){\rule[-0.200pt]{0.400pt}{34.930pt}}
\put(444.0,131.0){\rule[-0.200pt]{0.400pt}{35.171pt}}
\put(446.0,131.0){\rule[-0.200pt]{0.400pt}{35.412pt}}
\put(448.0,131.0){\rule[-0.200pt]{0.400pt}{35.653pt}}
\put(451.0,131.0){\rule[-0.200pt]{0.400pt}{35.653pt}}
\put(453.0,131.0){\rule[-0.200pt]{0.400pt}{35.653pt}}
\put(456.0,131.0){\rule[-0.200pt]{0.400pt}{35.894pt}}
\put(458.0,131.0){\rule[-0.200pt]{0.400pt}{35.894pt}}
\put(461.0,131.0){\rule[-0.200pt]{0.400pt}{35.894pt}}
\put(463.0,131.0){\rule[-0.200pt]{0.400pt}{35.894pt}}
\put(465.0,131.0){\rule[-0.200pt]{0.400pt}{36.135pt}}
\put(468.0,131.0){\rule[-0.200pt]{0.400pt}{36.376pt}}
\put(470.0,131.0){\rule[-0.200pt]{0.400pt}{36.376pt}}
\put(473.0,131.0){\rule[-0.200pt]{0.400pt}{36.376pt}}
\put(475.0,131.0){\rule[-0.200pt]{0.400pt}{36.376pt}}
\put(477.0,131.0){\rule[-0.200pt]{0.400pt}{36.617pt}}
\put(480.0,131.0){\rule[-0.200pt]{0.400pt}{36.617pt}}
\put(482.0,131.0){\rule[-0.200pt]{0.400pt}{36.617pt}}
\put(485.0,131.0){\rule[-0.200pt]{0.400pt}{36.617pt}}
\put(487.0,131.0){\rule[-0.200pt]{0.400pt}{36.858pt}}
\put(490.0,131.0){\rule[-0.200pt]{0.400pt}{36.858pt}}
\put(492.0,131.0){\rule[-0.200pt]{0.400pt}{36.858pt}}
\put(494.0,131.0){\rule[-0.200pt]{0.400pt}{37.099pt}}
\put(497.0,131.0){\rule[-0.200pt]{0.400pt}{37.099pt}}
\put(499.0,131.0){\rule[-0.200pt]{0.400pt}{37.099pt}}
\put(502.0,131.0){\rule[-0.200pt]{0.400pt}{37.339pt}}
\put(504.0,131.0){\rule[-0.200pt]{0.400pt}{37.339pt}}
\put(506.0,131.0){\rule[-0.200pt]{0.400pt}{37.339pt}}
\put(509.0,131.0){\rule[-0.200pt]{0.400pt}{37.339pt}}
\put(511.0,131.0){\rule[-0.200pt]{0.400pt}{37.339pt}}
\put(514.0,131.0){\rule[-0.200pt]{0.400pt}{37.339pt}}
\put(516.0,131.0){\rule[-0.200pt]{0.400pt}{37.580pt}}
\put(519.0,131.0){\rule[-0.200pt]{0.400pt}{37.580pt}}
\put(521.0,131.0){\rule[-0.200pt]{0.400pt}{37.580pt}}
\put(523.0,131.0){\rule[-0.200pt]{0.400pt}{37.580pt}}
\put(526.0,131.0){\rule[-0.200pt]{0.400pt}{37.821pt}}
\put(528.0,131.0){\rule[-0.200pt]{0.400pt}{37.821pt}}
\put(531.0,131.0){\rule[-0.200pt]{0.400pt}{38.062pt}}
\put(533.0,131.0){\rule[-0.200pt]{0.400pt}{38.303pt}}
\put(535.0,131.0){\rule[-0.200pt]{0.400pt}{38.303pt}}
\put(538.0,131.0){\rule[-0.200pt]{0.400pt}{38.303pt}}
\put(540.0,131.0){\rule[-0.200pt]{0.400pt}{38.303pt}}
\put(543.0,131.0){\rule[-0.200pt]{0.400pt}{38.303pt}}
\put(545.0,131.0){\rule[-0.200pt]{0.400pt}{38.303pt}}
\put(547.0,131.0){\rule[-0.200pt]{0.400pt}{38.544pt}}
\put(550.0,131.0){\rule[-0.200pt]{0.400pt}{38.544pt}}
\put(552.0,131.0){\rule[-0.200pt]{0.400pt}{38.544pt}}
\put(555.0,131.0){\rule[-0.200pt]{0.400pt}{39.026pt}}
\put(557.0,131.0){\rule[-0.200pt]{0.400pt}{39.026pt}}
\put(560.0,131.0){\rule[-0.200pt]{0.400pt}{39.267pt}}
\put(562.0,131.0){\rule[-0.200pt]{0.400pt}{39.267pt}}
\put(564.0,131.0){\rule[-0.200pt]{0.400pt}{39.508pt}}
\put(567.0,131.0){\rule[-0.200pt]{0.400pt}{39.508pt}}
\put(569.0,131.0){\rule[-0.200pt]{0.400pt}{39.508pt}}
\put(572.0,131.0){\rule[-0.200pt]{0.400pt}{39.748pt}}
\put(574.0,131.0){\rule[-0.200pt]{0.400pt}{39.748pt}}
\put(576.0,131.0){\rule[-0.200pt]{0.400pt}{39.748pt}}
\put(579.0,131.0){\rule[-0.200pt]{0.400pt}{39.989pt}}
\put(581.0,131.0){\rule[-0.200pt]{0.400pt}{39.989pt}}
\put(584.0,131.0){\rule[-0.200pt]{0.400pt}{39.989pt}}
\put(586.0,131.0){\rule[-0.200pt]{0.400pt}{39.989pt}}
\put(589.0,131.0){\rule[-0.200pt]{0.400pt}{40.230pt}}
\put(591.0,131.0){\rule[-0.200pt]{0.400pt}{40.230pt}}
\put(593.0,131.0){\rule[-0.200pt]{0.400pt}{40.230pt}}
\put(596.0,131.0){\rule[-0.200pt]{0.400pt}{40.230pt}}
\put(598.0,131.0){\rule[-0.200pt]{0.400pt}{40.230pt}}
\put(601.0,131.0){\rule[-0.200pt]{0.400pt}{40.471pt}}
\put(603.0,131.0){\rule[-0.200pt]{0.400pt}{40.471pt}}
\put(605.0,131.0){\rule[-0.200pt]{0.400pt}{40.712pt}}
\put(608.0,131.0){\rule[-0.200pt]{0.400pt}{40.712pt}}
\put(610.0,131.0){\rule[-0.200pt]{0.400pt}{40.712pt}}
\put(613.0,131.0){\rule[-0.200pt]{0.400pt}{40.712pt}}
\put(615.0,131.0){\rule[-0.200pt]{0.400pt}{40.712pt}}
\put(618.0,131.0){\rule[-0.200pt]{0.400pt}{40.953pt}}
\put(620.0,131.0){\rule[-0.200pt]{0.400pt}{41.194pt}}
\put(622.0,131.0){\rule[-0.200pt]{0.400pt}{41.194pt}}
\put(625.0,131.0){\rule[-0.200pt]{0.400pt}{41.435pt}}
\put(627.0,131.0){\rule[-0.200pt]{0.400pt}{41.435pt}}
\put(630.0,131.0){\rule[-0.200pt]{0.400pt}{41.435pt}}
\put(632.0,131.0){\rule[-0.200pt]{0.400pt}{41.676pt}}
\put(634.0,131.0){\rule[-0.200pt]{0.400pt}{41.676pt}}
\put(637.0,131.0){\rule[-0.200pt]{0.400pt}{41.676pt}}
\put(639.0,131.0){\rule[-0.200pt]{0.400pt}{41.676pt}}
\put(642.0,131.0){\rule[-0.200pt]{0.400pt}{41.676pt}}
\put(644.0,131.0){\rule[-0.200pt]{0.400pt}{41.676pt}}
\put(647.0,131.0){\rule[-0.200pt]{0.400pt}{41.676pt}}
\put(649.0,131.0){\rule[-0.200pt]{0.400pt}{41.917pt}}
\put(651.0,131.0){\rule[-0.200pt]{0.400pt}{41.917pt}}
\put(654.0,131.0){\rule[-0.200pt]{0.400pt}{41.917pt}}
\put(656.0,131.0){\rule[-0.200pt]{0.400pt}{42.157pt}}
\put(659.0,131.0){\rule[-0.200pt]{0.400pt}{42.157pt}}
\put(661.0,131.0){\rule[-0.200pt]{0.400pt}{42.157pt}}
\put(663.0,131.0){\rule[-0.200pt]{0.400pt}{42.157pt}}
\put(666.0,131.0){\rule[-0.200pt]{0.400pt}{42.157pt}}
\put(668.0,131.0){\rule[-0.200pt]{0.400pt}{42.398pt}}
\put(671.0,131.0){\rule[-0.200pt]{0.400pt}{42.639pt}}
\put(673.0,131.0){\rule[-0.200pt]{0.400pt}{42.639pt}}
\put(676.0,131.0){\rule[-0.200pt]{0.400pt}{42.880pt}}
\put(678.0,131.0){\rule[-0.200pt]{0.400pt}{42.880pt}}
\put(680.0,131.0){\rule[-0.200pt]{0.400pt}{43.121pt}}
\put(683.0,131.0){\rule[-0.200pt]{0.400pt}{43.121pt}}
\put(685.0,131.0){\rule[-0.200pt]{0.400pt}{43.121pt}}
\put(688.0,131.0){\rule[-0.200pt]{0.400pt}{43.121pt}}
\put(690.0,131.0){\rule[-0.200pt]{0.400pt}{43.121pt}}
\put(692.0,131.0){\rule[-0.200pt]{0.400pt}{43.362pt}}
\put(695.0,131.0){\rule[-0.200pt]{0.400pt}{43.362pt}}
\put(697.0,131.0){\rule[-0.200pt]{0.400pt}{43.362pt}}
\put(700.0,131.0){\rule[-0.200pt]{0.400pt}{43.603pt}}
\put(702.0,131.0){\rule[-0.200pt]{0.400pt}{43.603pt}}
\put(705.0,131.0){\rule[-0.200pt]{0.400pt}{43.844pt}}
\put(707.0,131.0){\rule[-0.200pt]{0.400pt}{43.844pt}}
\put(709.0,131.0){\rule[-0.200pt]{0.400pt}{43.844pt}}
\put(712.0,131.0){\rule[-0.200pt]{0.400pt}{43.844pt}}
\put(714.0,131.0){\rule[-0.200pt]{0.400pt}{43.844pt}}
\put(717.0,131.0){\rule[-0.200pt]{0.400pt}{44.085pt}}
\put(719.0,131.0){\rule[-0.200pt]{0.400pt}{44.085pt}}
\put(721.0,131.0){\rule[-0.200pt]{0.400pt}{44.326pt}}
\put(724.0,131.0){\rule[-0.200pt]{0.400pt}{44.326pt}}
\put(726.0,131.0){\rule[-0.200pt]{0.400pt}{44.566pt}}
\put(729.0,131.0){\rule[-0.200pt]{0.400pt}{44.807pt}}
\put(731.0,131.0){\rule[-0.200pt]{0.400pt}{45.048pt}}
\put(734.0,131.0){\rule[-0.200pt]{0.400pt}{45.048pt}}
\put(736.0,131.0){\rule[-0.200pt]{0.400pt}{45.048pt}}
\put(738.0,131.0){\rule[-0.200pt]{0.400pt}{45.048pt}}
\put(741.0,131.0){\rule[-0.200pt]{0.400pt}{45.048pt}}
\put(743.0,131.0){\rule[-0.200pt]{0.400pt}{45.289pt}}
\put(746.0,131.0){\rule[-0.200pt]{0.400pt}{45.289pt}}
\put(748.0,131.0){\rule[-0.200pt]{0.400pt}{45.289pt}}
\put(750.0,131.0){\rule[-0.200pt]{0.400pt}{45.289pt}}
\put(753.0,131.0){\rule[-0.200pt]{0.400pt}{45.289pt}}
\put(755.0,131.0){\rule[-0.200pt]{0.400pt}{45.289pt}}
\put(758.0,131.0){\rule[-0.200pt]{0.400pt}{45.530pt}}
\put(760.0,131.0){\rule[-0.200pt]{0.400pt}{45.530pt}}
\put(763.0,131.0){\rule[-0.200pt]{0.400pt}{45.530pt}}
\put(765.0,131.0){\rule[-0.200pt]{0.400pt}{45.771pt}}
\put(767.0,131.0){\rule[-0.200pt]{0.400pt}{45.771pt}}
\put(770.0,131.0){\rule[-0.200pt]{0.400pt}{46.012pt}}
\put(772.0,131.0){\rule[-0.200pt]{0.400pt}{46.012pt}}
\put(775.0,131.0){\rule[-0.200pt]{0.400pt}{46.012pt}}
\put(777.0,131.0){\rule[-0.200pt]{0.400pt}{46.012pt}}
\put(779.0,131.0){\rule[-0.200pt]{0.400pt}{46.253pt}}
\put(782.0,131.0){\rule[-0.200pt]{0.400pt}{46.253pt}}
\put(784.0,131.0){\rule[-0.200pt]{0.400pt}{46.494pt}}
\put(787.0,131.0){\rule[-0.200pt]{0.400pt}{46.494pt}}
\put(789.0,131.0){\rule[-0.200pt]{0.400pt}{46.735pt}}
\put(792.0,131.0){\rule[-0.200pt]{0.400pt}{46.735pt}}
\put(794.0,131.0){\rule[-0.200pt]{0.400pt}{46.975pt}}
\put(796.0,131.0){\rule[-0.200pt]{0.400pt}{46.975pt}}
\put(799.0,131.0){\rule[-0.200pt]{0.400pt}{47.216pt}}
\put(801.0,131.0){\rule[-0.200pt]{0.400pt}{47.216pt}}
\put(804.0,131.0){\rule[-0.200pt]{0.400pt}{47.216pt}}
\put(806.0,131.0){\rule[-0.200pt]{0.400pt}{47.457pt}}
\put(808.0,131.0){\rule[-0.200pt]{0.400pt}{47.698pt}}
\put(811.0,131.0){\rule[-0.200pt]{0.400pt}{47.698pt}}
\put(813.0,131.0){\rule[-0.200pt]{0.400pt}{47.698pt}}
\put(816.0,131.0){\rule[-0.200pt]{0.400pt}{47.939pt}}
\put(818.0,131.0){\rule[-0.200pt]{0.400pt}{47.939pt}}
\put(821.0,131.0){\rule[-0.200pt]{0.400pt}{48.180pt}}
\put(823.0,131.0){\rule[-0.200pt]{0.400pt}{48.180pt}}
\put(825.0,131.0){\rule[-0.200pt]{0.400pt}{48.180pt}}
\put(828.0,131.0){\rule[-0.200pt]{0.400pt}{48.180pt}}
\put(830.0,131.0){\rule[-0.200pt]{0.400pt}{48.662pt}}
\put(833.0,131.0){\rule[-0.200pt]{0.400pt}{48.662pt}}
\put(835.0,131.0){\rule[-0.200pt]{0.400pt}{48.662pt}}
\put(837.0,131.0){\rule[-0.200pt]{0.400pt}{48.903pt}}
\put(840.0,131.0){\rule[-0.200pt]{0.400pt}{49.144pt}}
\put(842.0,131.0){\rule[-0.200pt]{0.400pt}{49.144pt}}
\put(845.0,131.0){\rule[-0.200pt]{0.400pt}{49.384pt}}
\put(847.0,131.0){\rule[-0.200pt]{0.400pt}{49.625pt}}
\put(849.0,131.0){\rule[-0.200pt]{0.400pt}{49.866pt}}
\put(852.0,131.0){\rule[-0.200pt]{0.400pt}{50.107pt}}
\put(854.0,131.0){\rule[-0.200pt]{0.400pt}{50.107pt}}
\put(857.0,131.0){\rule[-0.200pt]{0.400pt}{50.107pt}}
\put(859.0,131.0){\rule[-0.200pt]{0.400pt}{50.589pt}}
\put(862.0,131.0){\rule[-0.200pt]{0.400pt}{50.830pt}}
\put(864.0,131.0){\rule[-0.200pt]{0.400pt}{51.071pt}}
\put(866.0,131.0){\rule[-0.200pt]{0.400pt}{51.553pt}}
\put(869.0,131.0){\rule[-0.200pt]{0.400pt}{51.553pt}}
\put(871.0,131.0){\rule[-0.200pt]{0.400pt}{51.553pt}}
\put(874.0,131.0){\rule[-0.200pt]{0.400pt}{51.793pt}}
\put(876.0,131.0){\rule[-0.200pt]{0.400pt}{51.793pt}}
\put(878.0,131.0){\rule[-0.200pt]{0.400pt}{51.793pt}}
\put(881.0,131.0){\rule[-0.200pt]{0.400pt}{52.034pt}}
\put(883.0,131.0){\rule[-0.200pt]{0.400pt}{52.034pt}}
\put(886.0,131.0){\rule[-0.200pt]{0.400pt}{52.275pt}}
\put(888.0,131.0){\rule[-0.200pt]{0.400pt}{52.516pt}}
\put(891.0,131.0){\rule[-0.200pt]{0.400pt}{53.239pt}}
\put(893.0,131.0){\rule[-0.200pt]{0.400pt}{53.239pt}}
\put(895.0,131.0){\rule[-0.200pt]{0.400pt}{53.239pt}}
\put(898.0,131.0){\rule[-0.200pt]{0.400pt}{53.239pt}}
\put(900.0,131.0){\rule[-0.200pt]{0.400pt}{53.480pt}}
\put(903.0,131.0){\rule[-0.200pt]{0.400pt}{53.480pt}}
\put(905.0,131.0){\rule[-0.200pt]{0.400pt}{53.721pt}}
\put(907.0,131.0){\rule[-0.200pt]{0.400pt}{53.721pt}}
\put(910.0,131.0){\rule[-0.200pt]{0.400pt}{53.721pt}}
\put(912.0,131.0){\rule[-0.200pt]{0.400pt}{53.962pt}}
\put(915.0,131.0){\rule[-0.200pt]{0.400pt}{53.962pt}}
\put(917.0,131.0){\rule[-0.200pt]{0.400pt}{53.962pt}}
\put(920.0,131.0){\rule[-0.200pt]{0.400pt}{53.962pt}}
\put(922.0,131.0){\rule[-0.200pt]{0.400pt}{54.925pt}}
\put(924.0,131.0){\rule[-0.200pt]{0.400pt}{54.925pt}}
\put(927.0,131.0){\rule[-0.200pt]{0.400pt}{55.648pt}}
\put(929.0,131.0){\rule[-0.200pt]{0.400pt}{55.889pt}}
\put(932.0,131.0){\rule[-0.200pt]{0.400pt}{56.130pt}}
\put(934.0,131.0){\rule[-0.200pt]{0.400pt}{56.130pt}}
\put(936.0,131.0){\rule[-0.200pt]{0.400pt}{56.371pt}}
\put(939.0,131.0){\rule[-0.200pt]{0.400pt}{56.852pt}}
\put(941.0,131.0){\rule[-0.200pt]{0.400pt}{56.852pt}}
\put(944.0,131.0){\rule[-0.200pt]{0.400pt}{57.093pt}}
\put(946.0,131.0){\rule[-0.200pt]{0.400pt}{57.093pt}}
\put(949.0,131.0){\rule[-0.200pt]{0.400pt}{57.575pt}}
\put(951.0,131.0){\rule[-0.200pt]{0.400pt}{58.298pt}}
\put(953.0,131.0){\rule[-0.200pt]{0.400pt}{59.743pt}}
\put(956.0,131.0){\rule[-0.200pt]{0.400pt}{59.743pt}}
\put(958.0,131.0){\rule[-0.200pt]{0.400pt}{59.743pt}}
\put(961.0,131.0){\rule[-0.200pt]{0.400pt}{59.743pt}}
\put(963.0,131.0){\rule[-0.200pt]{0.400pt}{60.466pt}}
\put(965.0,131.0){\rule[-0.200pt]{0.400pt}{60.707pt}}
\put(968.0,131.0){\rule[-0.200pt]{0.400pt}{60.948pt}}
\put(970.0,131.0){\rule[-0.200pt]{0.400pt}{60.948pt}}
\put(973.0,131.0){\rule[-0.200pt]{0.400pt}{60.948pt}}
\put(975.0,131.0){\rule[-0.200pt]{0.400pt}{61.189pt}}
\put(978.0,131.0){\rule[-0.200pt]{0.400pt}{61.429pt}}
\put(980.0,131.0){\rule[-0.200pt]{0.400pt}{61.429pt}}
\put(982.0,131.0){\rule[-0.200pt]{0.400pt}{62.393pt}}
\put(985.0,131.0){\rule[-0.200pt]{0.400pt}{62.634pt}}
\put(987.0,131.0){\rule[-0.200pt]{0.400pt}{62.634pt}}
\put(990.0,131.0){\rule[-0.200pt]{0.400pt}{63.598pt}}
\put(992.0,131.0){\rule[-0.200pt]{0.400pt}{64.079pt}}
\put(994.0,131.0){\rule[-0.200pt]{0.400pt}{64.320pt}}
\put(997.0,131.0){\rule[-0.200pt]{0.400pt}{64.561pt}}
\put(999.0,131.0){\rule[-0.200pt]{0.400pt}{64.802pt}}
\put(1002.0,131.0){\rule[-0.200pt]{0.400pt}{65.043pt}}
\put(1004.0,131.0){\rule[-0.200pt]{0.400pt}{66.007pt}}
\put(1007.0,131.0){\rule[-0.200pt]{0.400pt}{66.488pt}}
\put(1009.0,131.0){\rule[-0.200pt]{0.400pt}{66.488pt}}
\put(1011.0,131.0){\rule[-0.200pt]{0.400pt}{66.970pt}}
\put(1014.0,131.0){\rule[-0.200pt]{0.400pt}{67.211pt}}
\put(1016.0,131.0){\rule[-0.200pt]{0.400pt}{67.211pt}}
\put(1019.0,131.0){\rule[-0.200pt]{0.400pt}{67.211pt}}
\put(1021.0,131.0){\rule[-0.200pt]{0.400pt}{68.416pt}}
\put(1023.0,131.0){\rule[-0.200pt]{0.400pt}{68.656pt}}
\put(1026.0,131.0){\rule[-0.200pt]{0.400pt}{69.138pt}}
\put(1028.0,131.0){\rule[-0.200pt]{0.400pt}{70.102pt}}
\put(1031.0,131.0){\rule[-0.200pt]{0.400pt}{70.343pt}}
\put(1033.0,131.0){\rule[-0.200pt]{0.400pt}{70.584pt}}
\put(1036.0,131.0){\rule[-0.200pt]{0.400pt}{71.306pt}}
\put(1038.0,131.0){\rule[-0.200pt]{0.400pt}{71.306pt}}
\put(1040.0,131.0){\rule[-0.200pt]{0.400pt}{71.547pt}}
\put(1043.0,131.0){\rule[-0.200pt]{0.400pt}{71.788pt}}
\put(1045.0,131.0){\rule[-0.200pt]{0.400pt}{72.029pt}}
\put(1048.0,131.0){\rule[-0.200pt]{0.400pt}{72.511pt}}
\put(1050.0,131.0){\rule[-0.200pt]{0.400pt}{72.511pt}}
\put(1052.0,131.0){\rule[-0.200pt]{0.400pt}{72.993pt}}
\put(1055.0,131.0){\rule[-0.200pt]{0.400pt}{73.956pt}}
\put(1057.0,131.0){\rule[-0.200pt]{0.400pt}{74.197pt}}
\put(1060.0,131.0){\rule[-0.200pt]{0.400pt}{74.679pt}}
\put(1062.0,131.0){\rule[-0.200pt]{0.400pt}{74.679pt}}
\put(1065.0,131.0){\rule[-0.200pt]{0.400pt}{74.920pt}}
\put(1067.0,131.0){\rule[-0.200pt]{0.400pt}{75.883pt}}
\put(1069.0,131.0){\rule[-0.200pt]{0.400pt}{75.883pt}}
\put(1072.0,131.0){\rule[-0.200pt]{0.400pt}{76.124pt}}
\put(1074.0,131.0){\rule[-0.200pt]{0.400pt}{76.124pt}}
\put(1077.0,131.0){\rule[-0.200pt]{0.400pt}{76.365pt}}
\put(1079.0,131.0){\rule[-0.200pt]{0.400pt}{76.606pt}}
\put(1081.0,131.0){\rule[-0.200pt]{0.400pt}{78.052pt}}
\put(1084.0,131.0){\rule[-0.200pt]{0.400pt}{78.774pt}}
\put(1086.0,131.0){\rule[-0.200pt]{0.400pt}{79.497pt}}
\put(1089.0,131.0){\rule[-0.200pt]{0.400pt}{79.979pt}}
\put(1091.0,131.0){\rule[-0.200pt]{0.400pt}{84.074pt}}
\put(1094.0,131.0){\rule[-0.200pt]{0.400pt}{84.315pt}}
\put(1096.0,131.0){\rule[-0.200pt]{0.400pt}{84.315pt}}
\put(1098.0,131.0){\rule[-0.200pt]{0.400pt}{84.556pt}}
\put(1101.0,131.0){\rule[-0.200pt]{0.400pt}{85.279pt}}
\put(1103.0,131.0){\rule[-0.200pt]{0.400pt}{85.760pt}}
\put(1106.0,131.0){\rule[-0.200pt]{0.400pt}{86.242pt}}
\put(1108.0,131.0){\rule[-0.200pt]{0.400pt}{86.483pt}}
\put(1110.0,131.0){\rule[-0.200pt]{0.400pt}{87.447pt}}
\put(1113.0,131.0){\rule[-0.200pt]{0.400pt}{87.688pt}}
\put(1115.0,131.0){\rule[-0.200pt]{0.400pt}{88.169pt}}
\put(1118.0,131.0){\rule[-0.200pt]{0.400pt}{88.892pt}}
\put(1120.0,131.0){\rule[-0.200pt]{0.400pt}{90.337pt}}
\put(1123.0,131.0){\rule[-0.200pt]{0.400pt}{90.578pt}}
\put(1125.0,131.0){\rule[-0.200pt]{0.400pt}{91.060pt}}
\put(1127.0,131.0){\rule[-0.200pt]{0.400pt}{91.060pt}}
\put(1130.0,131.0){\rule[-0.200pt]{0.400pt}{91.783pt}}
\put(1132.0,131.0){\rule[-0.200pt]{0.400pt}{93.228pt}}
\put(1135.0,131.0){\rule[-0.200pt]{0.400pt}{94.192pt}}
\put(1137.0,131.0){\rule[-0.200pt]{0.400pt}{95.396pt}}
\put(1139.0,131.0){\rule[-0.200pt]{0.400pt}{95.396pt}}
\put(1142.0,131.0){\rule[-0.200pt]{0.400pt}{95.396pt}}
\put(1144.0,131.0){\rule[-0.200pt]{0.400pt}{96.119pt}}
\put(1147.0,131.0){\rule[-0.200pt]{0.400pt}{96.842pt}}
\put(1149.0,131.0){\rule[-0.200pt]{0.400pt}{97.324pt}}
\put(1151.0,131.0){\rule[-0.200pt]{0.400pt}{97.324pt}}
\put(1154.0,131.0){\rule[-0.200pt]{0.400pt}{97.564pt}}
\put(1156.0,131.0){\rule[-0.200pt]{0.400pt}{98.287pt}}
\put(1159.0,131.0){\rule[-0.200pt]{0.400pt}{99.010pt}}
\put(1161.0,131.0){\rule[-0.200pt]{0.400pt}{99.492pt}}
\put(1164.0,131.0){\rule[-0.200pt]{0.400pt}{100.455pt}}
\put(1166.0,131.0){\rule[-0.200pt]{0.400pt}{100.455pt}}
\put(1168.0,131.0){\rule[-0.200pt]{0.400pt}{100.937pt}}
\put(1171.0,131.0){\rule[-0.200pt]{0.400pt}{101.419pt}}
\put(1173.0,131.0){\rule[-0.200pt]{0.400pt}{101.660pt}}
\put(1176.0,131.0){\rule[-0.200pt]{0.400pt}{101.901pt}}
\put(1178.0,131.0){\rule[-0.200pt]{0.400pt}{101.901pt}}
\put(1180.0,131.0){\rule[-0.200pt]{0.400pt}{103.105pt}}
\put(1183.0,131.0){\rule[-0.200pt]{0.400pt}{103.346pt}}
\put(1185.0,131.0){\rule[-0.200pt]{0.400pt}{103.828pt}}
\put(1188.0,131.0){\rule[-0.200pt]{0.400pt}{103.828pt}}
\put(1190.0,131.0){\rule[-0.200pt]{0.400pt}{104.069pt}}
\put(1193.0,131.0){\rule[-0.200pt]{0.400pt}{105.273pt}}
\put(1195.0,131.0){\rule[-0.200pt]{0.400pt}{105.514pt}}
\put(1197.0,131.0){\rule[-0.200pt]{0.400pt}{105.996pt}}
\put(1200.0,131.0){\rule[-0.200pt]{0.400pt}{106.719pt}}
\put(1202.0,131.0){\rule[-0.200pt]{0.400pt}{106.719pt}}
\put(1205.0,131.0){\rule[-0.200pt]{0.400pt}{107.200pt}}
\put(1207.0,131.0){\rule[-0.200pt]{0.400pt}{107.441pt}}
\put(1209.0,131.0){\rule[-0.200pt]{0.400pt}{107.923pt}}
\put(1212.0,131.0){\rule[-0.200pt]{0.400pt}{107.923pt}}
\put(1214.0,131.0){\rule[-0.200pt]{0.400pt}{108.164pt}}
\put(1217.0,131.0){\rule[-0.200pt]{0.400pt}{108.164pt}}
\put(1219.0,131.0){\rule[-0.200pt]{0.400pt}{108.405pt}}
\put(1222.0,131.0){\rule[-0.200pt]{0.400pt}{108.405pt}}
\put(1224.0,131.0){\rule[-0.200pt]{0.400pt}{108.405pt}}
\put(1226.0,131.0){\rule[-0.200pt]{0.400pt}{108.646pt}}
\put(1229.0,131.0){\rule[-0.200pt]{0.400pt}{108.646pt}}
\put(1231.0,131.0){\rule[-0.200pt]{0.400pt}{108.887pt}}
\put(1234.0,131.0){\rule[-0.200pt]{0.400pt}{109.128pt}}
\put(1236.0,131.0){\rule[-0.200pt]{0.400pt}{109.369pt}}
\put(1238.0,131.0){\rule[-0.200pt]{0.400pt}{109.609pt}}
\put(1241.0,131.0){\rule[-0.200pt]{0.400pt}{109.609pt}}
\put(1243.0,131.0){\rule[-0.200pt]{0.400pt}{110.091pt}}
\put(1246.0,131.0){\rule[-0.200pt]{0.400pt}{110.091pt}}
\put(1248.0,131.0){\rule[-0.200pt]{0.400pt}{110.091pt}}
\put(1251.0,131.0){\rule[-0.200pt]{0.400pt}{110.091pt}}
\put(1253.0,131.0){\rule[-0.200pt]{0.400pt}{110.573pt}}
\put(1255.0,131.0){\rule[-0.200pt]{0.400pt}{110.573pt}}
\put(1258.0,131.0){\rule[-0.200pt]{0.400pt}{111.055pt}}
\put(1260.0,131.0){\rule[-0.200pt]{0.400pt}{111.055pt}}
\put(1263.0,131.0){\rule[-0.200pt]{0.400pt}{112.018pt}}
\put(1265.0,131.0){\rule[-0.200pt]{0.400pt}{112.259pt}}
\put(1267.0,131.0){\rule[-0.200pt]{0.400pt}{112.500pt}}
\put(1270.0,131.0){\rule[-0.200pt]{0.400pt}{112.500pt}}
\put(1272.0,131.0){\rule[-0.200pt]{0.400pt}{112.982pt}}
\put(1275.0,131.0){\rule[-0.200pt]{0.400pt}{112.982pt}}
\put(1277.0,131.0){\rule[-0.200pt]{0.400pt}{113.223pt}}
\put(1280.0,131.0){\rule[-0.200pt]{0.400pt}{115.632pt}}
\put(1282.0,131.0){\rule[-0.200pt]{0.400pt}{116.355pt}}
\put(1284.0,131.0){\rule[-0.200pt]{0.400pt}{116.596pt}}
\put(1287.0,131.0){\rule[-0.200pt]{0.400pt}{116.836pt}}
\put(1289.0,131.0){\rule[-0.200pt]{0.400pt}{117.800pt}}
\put(1292.0,131.0){\rule[-0.200pt]{0.400pt}{118.282pt}}
\put(1294.0,131.0){\rule[-0.200pt]{0.400pt}{118.523pt}}
\put(1296.0,131.0){\rule[-0.200pt]{0.400pt}{119.005pt}}
\put(1299.0,131.0){\rule[-0.200pt]{0.400pt}{119.968pt}}
\put(1301.0,131.0){\rule[-0.200pt]{0.400pt}{120.691pt}}
\put(1304.0,131.0){\rule[-0.200pt]{0.400pt}{122.377pt}}
\put(1306.0,131.0){\rule[-0.200pt]{0.400pt}{122.377pt}}
\put(1309.0,131.0){\rule[-0.200pt]{0.400pt}{122.859pt}}
\put(1311.0,131.0){\rule[-0.200pt]{0.400pt}{122.859pt}}
\put(1313.0,131.0){\rule[-0.200pt]{0.400pt}{123.100pt}}
\put(1316.0,131.0){\rule[-0.200pt]{0.400pt}{124.545pt}}
\put(1318.0,131.0){\rule[-0.200pt]{0.400pt}{126.713pt}}
\put(1321.0,131.0){\rule[-0.200pt]{0.400pt}{133.700pt}}
\put(1323.0,131.0){\rule[-0.200pt]{0.400pt}{139.240pt}}
\put(1325.0,131.0){\rule[-0.200pt]{0.400pt}{160.198pt}}
\put(231.0,131.0){\rule[-0.200pt]{0.400pt}{175.375pt}}
\put(231.0,131.0){\rule[-0.200pt]{291.007pt}{0.400pt}}
\put(1439.0,131.0){\rule[-0.200pt]{0.400pt}{175.375pt}}
\put(231.0,859.0){\rule[-0.200pt]{291.007pt}{0.400pt}}
\end{picture}
}
        \vspace{1cm}
        \resizebox{.9\linewidth}{!}{% GNUPLOT: LaTeX picture
\setlength{\unitlength}{0.240900pt}
\ifx\plotpoint\undefined\newsavebox{\plotpoint}\fi
\sbox{\plotpoint}{\rule[-0.200pt]{0.400pt}{0.400pt}}%
\begin{picture}(1500,900)(0,0)
\sbox{\plotpoint}{\rule[-0.200pt]{0.400pt}{0.400pt}}%
\put(171.0,131.0){\rule[-0.200pt]{4.818pt}{0.400pt}}
\put(151,131){\makebox(0,0)[r]{ 0}}
\put(1419.0,131.0){\rule[-0.200pt]{4.818pt}{0.400pt}}
\put(171.0,212.0){\rule[-0.200pt]{4.818pt}{0.400pt}}
\put(151,212){\makebox(0,0)[r]{ 0.5}}
\put(1419.0,212.0){\rule[-0.200pt]{4.818pt}{0.400pt}}
\put(171.0,293.0){\rule[-0.200pt]{4.818pt}{0.400pt}}
\put(151,293){\makebox(0,0)[r]{ 1}}
\put(1419.0,293.0){\rule[-0.200pt]{4.818pt}{0.400pt}}
\put(171.0,374.0){\rule[-0.200pt]{4.818pt}{0.400pt}}
\put(151,374){\makebox(0,0)[r]{ 1.5}}
\put(1419.0,374.0){\rule[-0.200pt]{4.818pt}{0.400pt}}
\put(171.0,455.0){\rule[-0.200pt]{4.818pt}{0.400pt}}
\put(151,455){\makebox(0,0)[r]{ 2}}
\put(1419.0,455.0){\rule[-0.200pt]{4.818pt}{0.400pt}}
\put(171.0,535.0){\rule[-0.200pt]{4.818pt}{0.400pt}}
\put(151,535){\makebox(0,0)[r]{ 2.5}}
\put(1419.0,535.0){\rule[-0.200pt]{4.818pt}{0.400pt}}
\put(171.0,616.0){\rule[-0.200pt]{4.818pt}{0.400pt}}
\put(151,616){\makebox(0,0)[r]{ 3}}
\put(1419.0,616.0){\rule[-0.200pt]{4.818pt}{0.400pt}}
\put(171.0,697.0){\rule[-0.200pt]{4.818pt}{0.400pt}}
\put(151,697){\makebox(0,0)[r]{ 3.5}}
\put(1419.0,697.0){\rule[-0.200pt]{4.818pt}{0.400pt}}
\put(171.0,778.0){\rule[-0.200pt]{4.818pt}{0.400pt}}
\put(151,778){\makebox(0,0)[r]{ 4}}
\put(1419.0,778.0){\rule[-0.200pt]{4.818pt}{0.400pt}}
\put(171.0,859.0){\rule[-0.200pt]{4.818pt}{0.400pt}}
\put(151,859){\makebox(0,0)[r]{ 4.5}}
\put(1419.0,859.0){\rule[-0.200pt]{4.818pt}{0.400pt}}
\put(171.0,131.0){\rule[-0.200pt]{0.400pt}{4.818pt}}
\put(171,90){\makebox(0,0){ 0}}
\put(171.0,839.0){\rule[-0.200pt]{0.400pt}{4.818pt}}
\put(382.0,131.0){\rule[-0.200pt]{0.400pt}{4.818pt}}
\put(382,90){\makebox(0,0){ 5}}
\put(382.0,839.0){\rule[-0.200pt]{0.400pt}{4.818pt}}
\put(594.0,131.0){\rule[-0.200pt]{0.400pt}{4.818pt}}
\put(594,90){\makebox(0,0){ 10}}
\put(594.0,839.0){\rule[-0.200pt]{0.400pt}{4.818pt}}
\put(805.0,131.0){\rule[-0.200pt]{0.400pt}{4.818pt}}
\put(805,90){\makebox(0,0){ 15}}
\put(805.0,839.0){\rule[-0.200pt]{0.400pt}{4.818pt}}
\put(1016.0,131.0){\rule[-0.200pt]{0.400pt}{4.818pt}}
\put(1016,90){\makebox(0,0){ 20}}
\put(1016.0,839.0){\rule[-0.200pt]{0.400pt}{4.818pt}}
\put(1228.0,131.0){\rule[-0.200pt]{0.400pt}{4.818pt}}
\put(1228,90){\makebox(0,0){ 25}}
\put(1228.0,839.0){\rule[-0.200pt]{0.400pt}{4.818pt}}
\put(1439.0,131.0){\rule[-0.200pt]{0.400pt}{4.818pt}}
\put(1439,90){\makebox(0,0){ 30}}
\put(1439.0,839.0){\rule[-0.200pt]{0.400pt}{4.818pt}}
\put(171.0,131.0){\rule[-0.200pt]{0.400pt}{175.375pt}}
\put(171.0,131.0){\rule[-0.200pt]{305.461pt}{0.400pt}}
\put(1439.0,131.0){\rule[-0.200pt]{0.400pt}{175.375pt}}
\put(171.0,859.0){\rule[-0.200pt]{305.461pt}{0.400pt}}
\put(30,495){\makebox(0,0){Speedup}}
\put(805,29){\makebox(0,0){Files of Polybench}}
\put(171.0,131.0){\rule[-0.200pt]{0.400pt}{8.431pt}}
\put(213.0,131.0){\rule[-0.200pt]{0.400pt}{11.804pt}}
\put(256.0,131.0){\rule[-0.200pt]{0.400pt}{12.527pt}}
\put(298.0,131.0){\rule[-0.200pt]{0.400pt}{13.490pt}}
\put(340.0,131.0){\rule[-0.200pt]{0.400pt}{14.936pt}}
\put(382.0,131.0){\rule[-0.200pt]{0.400pt}{16.140pt}}
\put(425.0,131.0){\rule[-0.200pt]{0.400pt}{18.067pt}}
\put(467.0,131.0){\rule[-0.200pt]{0.400pt}{21.681pt}}
\put(509.0,131.0){\rule[-0.200pt]{0.400pt}{22.404pt}}
\put(551.0,131.0){\rule[-0.200pt]{0.400pt}{25.054pt}}
\put(594.0,131.0){\rule[-0.200pt]{0.400pt}{25.054pt}}
\put(636.0,131.0){\rule[-0.200pt]{0.400pt}{25.054pt}}
\put(678.0,131.0){\rule[-0.200pt]{0.400pt}{25.294pt}}
\put(720.0,131.0){\rule[-0.200pt]{0.400pt}{26.258pt}}
\put(763.0,131.0){\rule[-0.200pt]{0.400pt}{27.222pt}}
\put(805.0,131.0){\rule[-0.200pt]{0.400pt}{34.208pt}}
\put(847.0,131.0){\rule[-0.200pt]{0.400pt}{35.412pt}}
\put(890.0,131.0){\rule[-0.200pt]{0.400pt}{36.376pt}}
\put(932.0,131.0){\rule[-0.200pt]{0.400pt}{39.748pt}}
\put(974.0,131.0){\rule[-0.200pt]{0.400pt}{40.712pt}}
\put(1016.0,131.0){\rule[-0.200pt]{0.400pt}{41.435pt}}
\put(1059.0,131.0){\rule[-0.200pt]{0.400pt}{43.121pt}}
\put(1101.0,131.0){\rule[-0.200pt]{0.400pt}{44.566pt}}
\put(1143.0,131.0){\rule[-0.200pt]{0.400pt}{45.530pt}}
\put(1185.0,131.0){\rule[-0.200pt]{0.400pt}{48.421pt}}
\put(1228.0,131.0){\rule[-0.200pt]{0.400pt}{50.348pt}}
\put(1270.0,131.0){\rule[-0.200pt]{0.400pt}{50.830pt}}
\put(1312.0,131.0){\rule[-0.200pt]{0.400pt}{55.407pt}}
\put(1354.0,131.0){\rule[-0.200pt]{0.400pt}{78.292pt}}
\put(1397.0,131.0){\rule[-0.200pt]{0.400pt}{89.615pt}}
\put(1439.0,131.0){\rule[-0.200pt]{0.400pt}{166.944pt}}
\put(171.0,131.0){\rule[-0.200pt]{0.400pt}{175.375pt}}
\put(171.0,131.0){\rule[-0.200pt]{305.461pt}{0.400pt}}
\put(1439.0,131.0){\rule[-0.200pt]{0.400pt}{175.375pt}}
\put(171.0,859.0){\rule[-0.200pt]{305.461pt}{0.400pt}}
\end{picture}
}
      \end{center}
  \end{columns}
\end{frame}

\frame{\frametitle{Conclusion and Future work}
  Conclusion
  \begin{itemize}
  \item New faster algorithm for SCoP detection
  \item Enable polyhedral optimization in industrial compilers
  \end{itemize}

  \vspace{1cm}
  
  Future Work
  \begin{itemize}
  \item SCoP detection to drive polyhedral optimization
  \item Use profile data to guide and select polyhedral transforms
  \end{itemize}
}


\end{document}
