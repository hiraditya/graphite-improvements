\title{Translating Gimple IR to/from Polyhedral representation}

\section{Goals}
\subsection{Gimple to isl}
\begin{enumerate}
    \item Translating Gimple IR to isl's polyhedral representation without modifying Gimple.
    \item Exposing cross basic block scalar dependencies to isl.
    \item Computing the iteration domain, building constraints/conditions of SCoP.
    \item Computing the schedule tree from the Gimple IR.
    \item Computing data references and theier dependencies.
\end{enumerate}

\subsection{Gimple to isl}
\begin{enumerate}
    \item Tagging parallel code from data dependence.
    \item Generating Gimple in SSA.
    \item Details about PHI nodes and their classification pertaining to codegen.
    \item Issues and design decisions e.g., ambiguities in codegen, bailing out in many cases, dealing with loop unrolled code.
    \item Discuss the cases where it might not be possible to generate code at all.
\end{enumerate}

\section{Internal parameters}
A parameter in SCoP is defined as any operand which is defined outside. These are first order parameters.
However, there are statements which only use parameters to define another operand. We call these as internal or
derived parameters.
IP = F (scop-parameters)
Since IPs can be derived from SESE parameters, they also do not require renaming when SESE is copied for polyhedral
transformations. While copying SESE we copy internal parameters at the beginning of the SCoP.
We maintain relative ordering of the internal parameters only as their ordering w.r.t. other
statements (which do not define internal parameters) does not affect the semantics of the program.
We could chose to copy IPs outside of the SCoP as well but we don't so as to preserve the original structure of the
program in case no polyhedral transformations were applied. This also preserves the live-range
of the original program and hence the register pressure.


\section{Limiting the number of computations done in ISL}
Newer releases of ISL have a way to bail out when total number of computations reach a limit.
This feature is very useful because this way we can limit the overhead in compile time. This also
allows greater control over optimization when loops are not too important. We provided a flag so
that users can tune it as required.


\section{Translating gimple IR to polyhedral representation without changing the original IR}
The gimple IR needs to be translated to polyhedral representation to be fed to isl.

This is done by first creating a copy of original SCoP and conditionally enabling one region.
Since the IR is in SSA form copying it requires a lot of book-keeping for scalars and PHIs.
Existing algorithm in graphite got around this problem by translating scalar to a one element array.
Since array is a data-reference, it does not require SSA renaming and book-keeping. This choice,
however, causes data references created in the original code. So even when there was no optimization done
IR was modified. Current algorithm removes this limitation by representing scalars as one element array only in the
polyhedral representation (virtually). This algorithm is more involved but avoids changes in the original code.

\subsection{Interesting scalars}
\begin{enumerate}
        \item Cross basic block scalar dependencies.
        \item Reads from outside of SCoPs
        \item Writes to outside of SCoP.
\end{enumerate}

\subsection{Build scattering polyhedrons from the program schedule}
Static and dynamic schedule are mapped in odd/even dimensions respectively.
The schedule can be improved by minimizing the number of dimensions as done in polly-llvm.

\section{Translating ISL polyhedral representation to gimple IR in SSA}
There are situations where it is not possible to map the resulting code back to the original.
\subsection {Translating PHIs}
\begin{enumerate}
  \item Cond PHI nodes. Finding the right direction from where SSA names/constants are coming. Issues with dominating
    preds.
  \item Loop PHI nodes. Finding the right basic block to insert loop PHIs.
  \item Loop closed-PHI nodes. Inserting closed-PHIs at all the merge points.
\end{enumerate}


\subsection {Disambiguating SSA names while mapping back}
\subsection {Issues while loop is unrolled}

\section{Measuring effectiveness of optimization}
Some way/heuristic of measuring 'amount' of optimization is reqruired because it is a costly optimization.
By having a metric would allow us to bail out early.
\begin{enumerate}
  \item Proximity
  \item Complexity in terms of code-size
\end{enumerate}

Bail out early based on following parameters.
\begin{enumerate}
  \item Optimization wasn't effective.
  \item Computation taking too long.
  \item Hotness of code.
  \item Code size.
  \item Number of parameters
\end{enumerate}

Essentially,

Number of ISL computations N = F (hotness, codesize, number of params, threshold of opt)
N increases with hotness, threshold of opt.
N decreases with codesize, number of params.

\section{Removing redundant dimensions from polyhedral represenation for faster compilation}

\section{Representing static schedule in ISL schedule node}

\section{Loop canonicalization}
Currently, we insert an empty basic block after each loop in a function. This is done so that
we can add PHIs to converge definitions from SCoPs from gimple and poly side. This changes
the IR but those changes are easily removed by subsequent passes. So it does not cause any
performance/code-size change in the generated code.


\section{Implementing loop fusion}

\section{Enabling graphite with profile guided optimization}
Although we have tried to redesign graphite for faster compile time, it still has some overhead.
Experimental results show an average of {} overhead in compile time.
Because of that, we would like to enable this for cases when compile time is not of much concern,
e.g. benchmarking, iterative compilation with feedback.

We have also enabled graphite to spend extra time on important loops. This can be done by tuning
isl-timeout programmatically based on the hotness of the loop. We believe expensive optimizations
can have the option to be demand driven so that compiler spends time on program portions which matters
most from the performance perspective. \ref{Duesterwald}


\section{Experimental Results}

\section{Conclusion and Future Work}
Iterative compilation with varying threshold of computations, and with profile info.


\references{}

S. Girbal, N. Vasilache, C. Bastoul, A. Cohen, D. Parello, M. Sigler, and O. Temam.
Semi-automatic composition of loop transformations for deep parallelism and memory
hierarchies. Intl. J. of Parallel Programming, 34(3):261–317, June 2006. Special issue on
Microgrids


U. Bondhugula, A. Hartono, J. Ramanujam, and P. Sadayappan. A practical automatic
polyhedral parallelization and locality optimization system. In ACM SIGPLAN Conf. on
Programming Languages Design and Implementation (PLDI’08), Tucson, AZ, USA, June
2008.

Trifunovic, Konrad, et al. "Graphite two years after: First lessons learned
from real-world polyhedral compilation."
GCC Research Opportunities Workshop (GROW'10). 2010.


Duesterwald, Evelyn, Rajiv Gupta, and Mary Lou Soffa. "Demand-driven computation of interprocedural data flow." Proceedings of the 22nd ACM SIGPLAN-SIGACT symposium on Principles of programming languages. ACM, 1995.
http://www.cs.ucr.edu/~gupta/research/Publications/Comp/popl95.pdf

